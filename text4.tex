\documentclass[a4paper,11pt]{article}
\usepackage[warn]{mathtext} 
\usepackage[russian]{babel} 
\usepackage[utf8x]{inputenc}
\usepackage{graphicx}
\usepackage[top=1cm,bottom=1.5cm,left=1cm,right=1cm]{geometry}
%\usepackage{pgfplots}
\usepackage{multirow}


\begin{document}
	\begin{flushleft}
		\textit{\textbf{Пункт 3.1 :} (Закон Ома)}
		\\
		\medskip
		\hangindent=1.5cm \hangafter=0 \noindent
		Снимите экспериментально и постройте графики зависимостей ${I=f(U)}$ при  ${R=const}$ и ${I=f(R)}$ при ${U=const}$.\\
		В расчетах использовались соответствующие математические соотношения между напряжением ${U}$, током ${I}$ и сопротивлением ${R}$ на участке цепи с сопротивлением :
		\begin{center}
			\begin{equation}
				I = \frac{U}{R}
			\end{equation}
			\begin{equation}
				U = I\cdot R
			\end{equation}
			\begin{equation}	
				R = \frac{U}{I}
			\end{equation}
		\end{center}
		где ${I}$ - ток, A; ${U}$ - напряжение, B; ${R}$ -  сопротивление, Ом.
		\\
		\medskip
		\textit{Результаты измерений : }
		\\
		\medskip
		\begin{center}
			\begin{tabular}{|c|c|c|c|c|c|c|c|}
\hline ${U, B}$ & 0 & 2 & 4 & 6 & 8 & 10 & 12 \\ 
\hline ${I}$, мА при ${R=100}$ Ом & 0 & 39,8 & 59,8 & 79,5 & 83,2 & 98,7 & 118,4 \\ 
\hline ${I}$, мА при ${R=150}$ Ом & 0 & 26,2 & 39,3 & 45,7 & 53 & 65,4 & 78,6 \\ 
\hline ${I}$, мА при ${R=330}$ Ом & 0 & 6 & 12 & 18 & 23,9 & 30 & 36 \\ 
\hline 
\end{tabular}
\\
\medskip
\end{center}
		\textit{График зависимости ${U(I)}$ для разных сопротивлений :}
		\\
		\medskip
		\begin{center}
			\includegraphics[scale=0.75]{gnuplot_p1_g1.png} 
		\end{center}
		\newpage
		Для построения кривых ${I=f(R)}$ измерим токи, имеющие место при напряжениях 4 В, 8 В и 12 В, в зависимости от соответствующих напряжений (100, 150, 220, 330, 470, 680 и 1000 Ом).\\
		\textit{Результаты измерений :}
		\begin{center}
			\begin{tabular}{|c|c|c|c|c|c|c|c|}
\hline ${R}$, Ом & 100 & 150 & 220 & 330 & 470 & 680 & 1000 \\ 
\hline ${I}$, при ${U=12 }$ В & 118,7 & 78,6 & 54 & 36 & 24,6 & 16,8 & 12 \\ 
\hline ${I}$, при ${U=8 }$ В & 79,4 & 50 & 36,1 & 24 & 16,4 & 11,2 & 8 \\ 
\hline ${I}$, при ${U=4 }$ В & 39,7 & 26,3 & 18,3 & 12 & 8,1 & 5,6 & 4 \\ 
\hline 
\end{tabular}
		\\
		\medskip 
		\end{center}
		\textit{График зависимости ${I(R)}$ для разных значений напряжения :}
		\\
		\medskip
		\begin{center}
			\includegraphics[scale=0.75]{plot2.png} 
		\end{center}
	\end{flushleft}
	\begin{flushleft}	
	\textit{\textbf{Пункт 4.2 :} (Линейные резисторы)}
	\\
	\medskip
	\hangindent=1.5cm \hangafter=0 \noindent
	Резистор называют \textit{\textbf{линейным}}, когда ток в нем изменяется пропорционально приложенному напряжению	то есть функция ${I=f(U)}$ прямолинейна.
	\\
	\medskip
	\textit{Результаты измерений (для схемы 4.2.1) :}
	\begin{center}
	\begin{tabular}{|c|c|p{1cm}|p{1cm}|p{1cm}|p{1cm}|p{1cm}|p{1cm}|}
	\hline
\multicolumn{2}{|c|}
  {${U}$, В}  &  2  &  4  &  6  &  8  &  10  &  12  \\\hline
\multirow{2}{*}{${R}=100$ Ом}
      & ${I}$, мА  & 0,02 & 0,04 & 0,07 & 0,09 & 0,12 & 0,15 \\
      & ${P}$, мВт & 0,04 & 0,16 & 0,42 & 0,72 & 1,2  & 1,8 \\\hline
\multirow{2}{*}{${R=150}$ Ом}      
      & ${I}$, мА  & 0,014 & 0,03 & 0,04 & 0,05 & 0,06 & 0,07 \\
      & ${P}$, мВт & 0,028 & 0,12 & 0,24 & 0,4 & 0,6 & 0,84 \\\hline
 \multirow{2}{*}{${R=330}$ Ом}
 	  & ${I}$, мА  & 0,006 & 0,012 & 0,018 & 0,2 & 0,2 & 0,2 \\
      & ${P}$, мВт & 0,012 & 0,048 & 0,108 & 1,6 & 2  &  2,4 \\	     
\hline
\end{tabular}
	\end{center}
	\newpage
	\textit{График зависимости ${U(I)}$ для разных резисторов :}
	\begin{center}
		\includegraphics[scale=0.78]{plot_3_res.png} 
	\end{center}
	Вычислим численное значение силы тока ${I}$, при котором в резисторе рассеивается ${P}$, равная 2 Вт, используя формулу :
	\begin{center}
		\begin{equation}
			I=\sqrt{\frac{P}{R}}			
		\end{equation}
	\end{center}
	\begin{list}{\labelitemi}{\leftmargin=3cm}
		\item Для ${R=100}$ Ом : ${I=\sqrt{\frac{2}{100}}=\sqrt{0.02}=0.1414214  }=0.14$ A
		\item Для ${R=150}$ Ом : ${I=\sqrt{\frac{2}{150}}=\sqrt{0.013}=0.1154701=0.12}$ A
		\item Для ${R=330}$ Ом : ${I=\sqrt{\frac{2}{330}}=\sqrt{0.0060606}= 0.0778499}=0.078$ A
	\end{list}
	Линия постоянной мощности указана на предыдущем графике и найдена подстановкой значений напряжения ${U}$  в формулу ${I=\frac{P}{U}}$ :
	\begin{center}
		\begin{tabular}{|p{1cm}|p{1cm}|p{1cm}|p{1cm}|p{1cm}|p{1cm}|p{1cm}|}
			\hline ${U}$, В & 2 & 4 & 6 & 8 & 10 & 12 \\ 
			\hline ${I}$, А & 1 & 0,5 & 0,34 & 0,25 & 0,2 &  0,167 \\ 
			\hline 
		\end{tabular} 
	\end{center}
	\newpage
	\textit{Линия мощности ${P = 2}$ Вт :}
	\begin{center}
		\includegraphics[scale=0.78]{Power.png}
	\end{center}		 
	\end{flushleft}
	\begin{flushleft}
	\textit{\textbf{Пункт 4.7 :} (Последовательное подключение резисторов)}
	\\
	\medskip
	\hangindent=1.5cm \hangafter=0 \noindent
	При последовательном соединении резисторов ток, проходящий	через каждый резистор цепи остается постоянной величиной, которая определяется приложенным напряжением ${U}$ и суммарным сопротивлением ${R_o}$ :
	\begin{equation}
		I=\frac{U}{R_o}
	\end{equation}
	\begin{equation}
		R_o = \sum_{i=1}^n R_i		
	\end{equation}
	На каждый отдельный резистор при этом приходится некоторое частичное напряжение.\\
	\medskip
		Сумма частичных напряжений в соответствии с 2-ым законом Кирхгофа равна одному приложенному напряжению :
		\begin{equation}
			U = \sum_{k=1}^n I \cdot R_k
		\end{equation}
	Измерим для схемы \textit{4.7.2} значения силы тока	для участков цепи \textbf{A-B, C-D, E-F}, и \textbf{G-H}.Измерим также значения падения напряжения между точками \textbf{B-C}, \textbf{D-E}, \textbf{F-G} а также полное напряжение между точкам \textbf{B-G}.
	\newpage
	Результаты измерений для схемы цепи \textit{4.7.2} :
	\begin{center}
		\begin{tabular}{|c|c|c|c|c|c|c|c|}
\hline 
\multicolumn{4}{|c|}{\textbf{Ток, мА}} & \multicolumn{3}{c|}{\textbf{Падения напряжения, В}} & {\textbf{Полное напряжение, В}}\tabularnewline
\hline 
\multicolumn{4}{|c|}{Точки цепи} & \multicolumn{3}{c|}{Точки цепи} & {Точки цепи}\tabularnewline
\hline 
\textbf{A-B} & \textbf{C-D} & \textbf{E-F} & \textbf{G-H} & \textbf{B-C} & \textbf{D-E} & \textbf{F-G} & \textbf{B-G}\tabularnewline
\hline 
12,6 & 12,3 & 12,5 & 12,4 & 1,23 & 2,73 & 6,03 & 8,76\tabularnewline
\hline 
\end{tabular}
\\
\medskip 	
	\end{center}
	Рассчитаем значения напряжения ${R}$ на всех участках цепи :
	\begin{center}
		\begin{tabular}{|c|c|c|c|}
\hline ${R_{BC}}$,  Ом & ${R_{DE}}$,  Ом & ${R_{FG}}$,  Ом & ${R_O}$,  Ом \\ 
\hline  0,0977 & 0,223 & 0,466 & 0,706 \\ 
\hline 
\end{tabular}
	\\
	\medskip 
	\end{center}
		С учетом погрешности измерений получаем :
	\begin{center}
		${R_O = R_{BC}+R_{DE}+R_{FG}}$\\
		${0,706 \approx 0,0977 + 0,223 + 0,466 = 0,7467}$ 
	\end{center}
%	\textit{Вопросы :}
%	\begin{list}{\labelitemi}{\leftmargin=3cm}	
%		\item 	
%	\end{list}}		
	\end{flushleft}
	\newpage 
		\begin{flushleft}
	\textit{\textbf{Пункт 4.8 :} (Параллельное подключение резисторов)}
	\\
	\medskip
	\hangindent=1.5cm \hangafter=0 \noindent
	Если резисторы и другие виды нагрузки соединены параллельно, то все эти элементы находятся под одинаковым напряжением.
	\\
	\medskip
	В каждом узле такой цепи протекает свой ток.Сумма токов всех ветвей в соответствии с 1-ым законом Кирхгофа равна полному току цепи :
	\begin{equation}
		I_o = \sum_{k=1}^n I_k 		
	\end{equation}
	, где ${n}$  - число ветвей цепи, содержащих нагрузку.
	\\
	\medskip
	Величина силы тока ${I}$ зависит от приложенного напряжения и сопротивления данной ветви цепи :
	\begin{center}
		${I_k = \frac{U}{R_k}}$
	\end{center}
	Ток в неразветвленной части цепи зависит от приложенного напряжения и эквивалентного сопротивления цепи :
	\begin{center}
		${I_{неразв} = \frac{U}{R_{экв}}}$
	\end{center}
	Выражение для вычисления эквивалентного сопротивления цепи :
	\begin{equation}
		R_{экв}=\frac{1}{\sum_{k=1}^n \frac{1}{R_k}}	
	\end{equation}
	Для цепи с 2 параллельно соединенными резисторами :
	\begin{equation}
		R_{экв}=\frac{R_1 \cdot R_2}{R_1 + R_2} 
	\end{equation} 
	Измерим для схемы \textit{4.8.1} значения силы тока	для участков цепи \textbf{A-B, C-D, E-F, G-H, L-K}.Измерим также значения напряжения на резисторах между ${R_1}$, ${R_2}$ и ${R_3}$ точками \textbf{D-K}, \textbf{F-K}, и \textbf{H-K} а также значение полной силы тока между точкам \textbf{A-B} и \textbf{L-K}.\\
	\medskip
	Результаты измерений для схемы цепи \textit{4.8.1} :
	\begin{center}
		\begin{tabular}{|c|c|c|c|c|c|c|c|}
\hline 
\multicolumn{3}{|c|}{\textbf{Напряжения, В}} & \multicolumn{3}{c|}{\textbf{Токи ветвей, мА}} & \multicolumn{2}{c|}{\textbf{Полный ток цепи, мА}}\tabularnewline
\hline 
\multicolumn{3}{|c|}{Точки измерения} & \multicolumn{3}{c|}{Точки измерения} & \multicolumn{2}{c|}{Точки измерения}\tabularnewline
\hline 
\textbf{D-K(R1)} & \textbf{F-K(R2}) & \textbf{H-K(R3)} & \textbf{C-D} & \textbf{E-F} & \textbf{G-H} & \textbf{A-B} & \textbf{L-K}\tabularnewline
\hline 
 9,9 & 9,98 & 9,98 & 0,1 & 0,04 & 0,02 & 0,14 & 0,16 \tabularnewline
\hline 
\end{tabular}
	\end{center}
	Рассчитаем значения напряжения ${R}$ на всех участках цепи (используя закон Ома ${R=\frac{U}{I}}$) :
		\begin{center}
		\begin{tabular}{|c|c|c|c|}
\hline ${R_{1}}$,  Ом & ${R_{2}}$,  Ом & ${R_{3}}$,  Ом & ${R_{полн}}$,  Ом \\ 
\hline  99  & 249,5 & 499 & 848 \\ 
\hline 
\end{tabular}
	\\
	\medskip 
	\end{center}
	Проверим выражение : ${\frac{1}{R_{полн}}} = \frac{1}{R_1} + \frac{1}{R_2} + \frac{1}{R_3}$ c (учетом погрешности) :
	\begin{center}
		${\frac{1}{848} \approx \frac{1}{99} + \frac{1}{249,5} + \frac{1}{499} }$
	\end{center}
	\textit{Вопросы : }\\
		\begin{list}{\labelitemi}{\leftmargin=3cm}	
			\item Полное сопротивление цепи с параллельным соединением резисторов равно 848 Ом.	 	
	        \item Токи ветвей обратно пропорциональные величинам сопротивлений этих параллельных ветвей.
	     \end{list}   	
	\end{flushleft}
	\newpage
	\begin{flushleft}
		\textit{\textbf{Пункт 4.9 } (Цепи со смешанной последовательно-параллельным соединением резисторов)}
		\\
		\medskip
		\hangindent=1.5cm \hangafter=0 \noindent
		Подадим на вход схемы \textit{4.9.1} и измерим силу тока ${I}$ во всех точках \textbf{A-B}, \textbf{C-D}, \textbf{E-F}. а также напряжение.\\
		Результаты измерений :
		\begin{center}
				\begin{tabular}{|c|c|c|c|c|c|c|c|}
\hline ${I_1}$, мА & ${I_2}$, мА & ${I_3}$, мА & ${U_{GH}}$ & ${U_{GM}}$ & ${U_{MA}}$ & ${U_{GA}}$ & ${U_{BH}}$ \\ 
\hline 27,7 & 18,9 & 8,7 & 15 & 6 & 2,8 & 8,6 & 5,2 \\ 
\hline 
\end{tabular} 
		\end{center} 
		Рассчитаем соответствующие значения сопротивления по закону Ома (${R=\frac{U}{I}}$):
		\begin{center}
			\begin{tabular}{|c|c|c|c|c|c|c|c|}
\hline  & ${R_1}$, Ом & ${R_2}$, Ом & ${R_3}$, Ом & ${R_4}$, Ом & ${R_{12}}$, Ом & ${R_{34}}$, Ом & ${R_{полн}}$, Ом \\ 
\hline Измеренные & 0,22 & 0,1 & 0,34 & 0,7 & 0,31 & 0,23 & 0,6 \\ 
\hline Рассчитанные & 0,2167 & 0,1011 &  0,328 & 0,7126 & 0,3178 & 0,2246 & 0,5424 \\ 
\hline Погрешность & 0,0033 & 0,0011 & 0,012 & 0,0126 & 0,0078 & 0,0054 & 0,0576 \\ 
\hline 
\end{tabular} 
		\end{center}
	где рассчитанные значения ${R_{12}}$ и ${R_{34}}$ вычисляются как :
	\begin{center}
		${R_{12}= R_1 + R_2}$ \\\medskip
		${R_{34}= \frac{R_3*R_4}{R_3+R_4}}$	
	\end{center}
	Рассчитаем расхождения результатов :
	\begin{list}{\labelitemi}{\leftmargin=3cm}
		\item Для ${R_1}$ = ${\frac{0,0033}{0,2167}}$ ${\cdot}$ 100  = 21,34 \%
		\item Для ${R_2}$ = ${\frac{0,0011}{0,1011}}$ ${\cdot}$ 100  = 10 \%
		\item Для ${R_3}$ = ${\frac{0,012}{0,328}}$ ${\cdot}$ 100  = 31,6 \%
		\item Для ${R_4}$ = ${\frac{0,0126}{0,7126}}$ ${\cdot}$ 100  = 70 \%
	\end{list}
	\end{flushleft}
			
\end{document}