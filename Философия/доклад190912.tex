\documentclass[a4paper,12pt]{article}
\usepackage[warn]{mathtext} 
\usepackage[russian]{babel} 
\usepackage[utf8x]{inputenc}
\usepackage{graphicx}
\usepackage[top=1cm,bottom=1.5cm,left=1cm,right=1cm]{geometry}


\begin{document}
Наука как особая форма мировоззрения 

Мировоззрение - система взглядов , понятий и представлений об окружающем мире. М. в широком смысле слова включает в себя совокупность всех взглядов человека на окружающий мир: философские, общественно-политические, этические, эстетические, естественнонаучные воззрения и т.д. Основное ядро всякого М. (М. в более узком смысле слова) составляют философские взгляды. Главным вопросом М. является основной вопрос философии. В зависимости от его решения различаются два основных вида мировоззрения: материалистическое и идеалистическое. М. является отражением общественного бытия и зависит от уровня человеческих знаний, достигнутых в данный исторический период, а также от общественного строя.

НАУКА — особый вид познавательной деятельности, нацеленный на выработку объективных, системно организованных и обоснованных знаний о мире. Социальный институт, обеспечивающий функционирование научной познавательной деятельности.
    Как вид познания наука взаимодействует с др. его видами: обыденным, художественным, религиозно-мифологическим, философским. Возникает из потребностей практики и особым способом регулирует ее. Наука ставит своей целью выявить сущностные связи (законы), в соответствии с которыми объекты могут преобразовываться в человеческой деятельности. Поскольку в деятельности могут преобразовываться любые объекты — фрагменты природы, социальные подсистемы и общество в целом, состояния человеческого сознания и т. п., постольку все они могут стать предметами научного исследования. Наука изучает их как объекты, функционирующие и развивающиеся по своим естественным законам. Она может Изучать и человека как субъекта деятельности, но тоже в качестве особого объекта.
    Предметный и объективный способ рассмотрения мира, характерный для науки, отличает ее от иных способов познания. Напр., в искусстве освоение действительности всегда происходит как своеобразная склейка субъективного и объективного, когда любое воспроизведение событий или состояний природы и социальной жизни предполагает их эмоциональную оценку. Художественный образ всегда выступает как единство общего и единичного, рационального и эмоционального. Научные же понятия — это рациональное, выделяющее общее и существенное в мире объектов.
    Отражая мир в его объективности, наука дает лишь один из срезов многообразия человеческого мира. Поэтому она не исчерпывает собой всей культуры, а составляет лишь одну из сфер, которая взаимодействует с др. сферами культурного творчества — моралью, религией, философией, искусством и т. д. Признак предметности и объективности знания является важнейшей характеристикой науки, но он еще недостаточен для определения ее специфики, поскольку отдельные объективные и предметные знания может давать и обыденное познание. Но в отличие от него наука не ограничивается изучением только тех объектов, их свойств и отношений, которые в принципе могут быть освоены в практике соответствующей исторической эпохи. Она способна выходить за рамки каждого исторически определенного типа практики и открывать для человечества новые предметные миры, которые могут стать объектами массового практического освоения лишь на будущих этапах развития цивилизации. Лейбниц характеризовал математику как науку о возможных мирах. В принципе эту характеристику можно отнести к любой фундаментальной науке. Электромагнитные волны, ядерные реакции, когерентные излучения атомов были вначале открыты в науке, и в этих открытиях потенциально был заложен принципиально новый уровень технологического развития цивилизации, который реализовался значительно позднее (техника электродвигателей и электрогенераторов, радио- и телеаппаратура, лазеры и атомные электростанции и т. д.). Постоянное стремление науки к расширению поля изучаемых объектов, безотносительно к сегодняшним возможностям их массового практического освоения, выступает тем системообразующим признаком, который обосновывает др. характеристики науки, отличающие ее от обыденного познания. Прежде всего — это отличие по их продуктам (результатам). Обыденное познание создает конгломерат знаний, сведений, предписаний и верований, лишь отдельные фрагменты которого связаны между собой. Истинность знаний проверяется здесь непосредственно в наличной практике, т. к. знания строятся относительно объектов, которые включены в процессы производства и наличного социального опыта. Но поскольку наука постоянно выходит за эти рамки, она лишь частично может опереться на наличные формы массового практического освоения объектов. Ей нужна особая практика, с помощью которой проверяется истинность ее знаний. Такой практикой становится научный эксперимент. Часть знаний непосредственно проверяется в эксперименте. Остальные связываются между собой логическими связями, что обеспечивает перенос истинности с одного высказывания на другое. В итоге возникают присущие науке характеристики ее знаний — их системная организация, обоснованность и доказанность.

— особый вид познавательной деятельности, направленный на выработку объективных, системно организованных и обоснованных знаний о мире. Взаимодействует с др. видами познавательной деятельности: обыденным, художественным, религиозным, мифологическим, филос. постижением мира. 
Как и все виды познания, Н. возникла из потребностей практики и особым способом регулирует ее. Н. ставит своей целью выявить сущностные связи (законы), в соответствии с которыми объекты могут преобразовываться в процессе человеческой деятельности. 
Любые объекты, допускающие преобразование человеком, — фрагменты природы, социальные подсистемы или общество в целом, состояния человеческого сознания и т.п. — могут стать предметами научного исследования. Н. изучает их как объекты, функционирующие и развивающиеся по своим естественным законам. Она может изучать и человека как субъекта деятельности, но также в качестве особого объекта. 
Предметный и объективный способ рассмотрения мира, характерный для Н., отличает ее от иных способов познания. Напр., в искусстве освоение действительности всегда происходит как своеобразная склейка субъективного и объективного, когда любое воспроизведение событий или состояний природы и социальной жизни предполагает их эмоциональную оценку. Художественный образ всегда является единством общего и единичного, рационального и эмоционального. Научные же понятия — это рациональное, выделяющее общее и существенное в мире объектов. 
Отражая мир в его объективности, А. дает лишь один из срезов многообразия человеческого мира. Поэтому она не исчерпывает собой всей культуры, а составляет лишь одну из сфер, которая взаимодействует с др. сферами культурного творчества — моралью, религией, философией, искусством и т.д. 
Признак предметности и объективности знания является важнейшей характеристикой Н., но он еще недостаточен для определения ее специфики, поскольку отдельные объективные и предметные знания может давать и обыденное познание. В отличие от него Н. не ограничивается изучением только тех объектов, их свойств и отношений, которые в принципе могут быть освоены в практике соответствующей исторической эпохи. Она способна выходить за рамки каждого исторически определенного типа практики и открывать для человечества новые предметные миры, которые могут стать объектами массового практического освоения лишь на будущих этапах развития цивилизации. В свое время Г.В. Лейбниц характеризовал математику как Н. о возможных мирах. В принципе эту характеристику можно отнести к любой фундаментальной Н. Электромагнитные волны, ядерные реакции, когерентные излучения атомов вначале были открыты в физике, и в этих открытиях потенциально был заложен принципиально новый уровень технологического развития цивилизации, реализовавшийся значительно позднее (техника электродвигателей и электрогенераторов, радио- и телеаппаратура, лазеры, атомные электростанции и т.д.). 
Постоянное стремление Н. к расширению поля изучаемых объектов, безотносительно к сегодняшним возможностям их массового практического освоения, выступает тем системообразующим признаком, который обосновывает др. характеристики Н., отличающие ее от обыденного познания. 
Прежде всего — это отличие по продуктам (результатам). Обыденное познание создает конгломерат знаний, сведений, предписаний и верований, лишь отдельные фрагменты которого связаны между собой. Истинность знаний проверяется здесь непосредственно наличной практикой, т.к. знания строятся относительно объектов, включенных в процессы производства и наличного социального опыта. 
Но поскольку Н. постоянно выходит за эти рамки, она лишь частично может опереться на наличные формы массового практического освоения объектов. Ей нужна особая практика, с помощью которой проверяется истинность ее знаний. Такой практикой становится научный эксперимент. Часть знаний непосредственно проверяется в эксперименте. Остальные объединяются логическими связями, что обеспечивает перенос истинности с одного высказывания на другое. В итоге возникают такие присущие Н. характеристики, как системная организация ее знаний, их обоснованность и доказанность. 
Далее, Н., в отличие от обыденного познания, предполагает применение особых средств и методов деятельности. Она не может ограничиться использованием только обыденного языка и тех орудий, которые применяются в производстве и повседневной практике. Ей необходимы особые средства деятельности — специальный язык (эмпирический и теоретический) и особые приборные комплексы. Именно постоянное развитие этих средств обеспечивает исследование все новых объектов, в т.ч. и тех, которые выходят за рамки возможностей наличной производственной и социальной практики. Этим же вызваны потребности Н. в постоянной разработке специальных методов, обеспечивающих освоение новых объектов безотносительно к возможностям их сегодняшнего практического освоения. 
Метод в Н. часто служит условием фиксации и воспроизводства объекта исследования; наряду со знанием об объектах Н. систематически развивает знания о методах. 
Наконец, существуют специфические особенности субъекта научной деятельности. Субъект обыденного познания формируется в самом процессе социализации. Для Н. же этого недостаточно: требуется особое обучение познающего субъекта, которое обеспечивает его умение применять свойственные Н. средства и методы при решении ее задач и проблем. Систематические занятия Н. предполагают также усвоение особой системы ценностей. Фундаментом выступают ценностные установки на поиск истины и на постоянное наращивание истинного знания. На базе этих установок исторически развивается система идеалов и норм научного исследования. Эти ценностные установки составляют основание этики Н. Система идеалов Н. запрещает умышленное искажение истины в угоду тем или иным социальным целям, система норм научного исследования требует постоянной инновационной деятельности и вводит запрет на плагиат. 
Фундаментальные ценностные установки соответствуют двум фундаментальным и определяющим признакам Н.: предметности и объективности научного познания и ее интенции на изучение все новых объектов, безотносительно к наличным возможностям их массового практического освоения. 
В развитии научного знания можно выделить преднауку и Н. в собственном смысле слова. На начальной стадии зарождающаяся Н. еще не выходит за рамки наличной практики. Она моделирует изменение объектов, включенных в практическую деятельность, предсказывая их возможные состояния. Реальные объекты замещаются в познании идеальными объектами и выступают как абстракции, которыми оперирует мышление. Их связи и отношения, операции с ними также черпаются из практики, выступая как схемы практических действий. Такой характер имели, напр., геометрические знания древних египтян. Первые геометрические фигуры были моделями земельных участков; операции разметки участка с помощью мерной веревки, закрепленной на конце колышком, что позволяло проводить дуги, были схематизированы и стали способом построения геометрических фигур с помощью циркуля и линейки. 
Переход от преднауки к собственно Н. вызван новым способом формирования идеальных объектов и их связей, моделирующих практику. В развитой Н. они не только черпаются непосредственно из практики, но преимущественно создаются в качестве абстракций, на основе ранее введенных идеальных объектов. Построенные из их связей модели выступают в качестве гипотез, которые затем, получив обоснование, превращаются в теоретические схемы изучаемой предметной области. Так возникает особое движение в сфере развивающегося теоретического знания, которое начинает строить модели изучаемой реальности как бы сверху по отношению к практике с их последующей прямой или косвенной практической проверкой. 
Исторически первой осуществила переход к собственно научному познанию мира математика. Затем способ теоретического познания, основанный на движении мысли в поле теоретически идеальных объектов с последующей экспериментальной проверкой гипотез, утвердился в естествознании. Третьей вехой в развитии Н. стало формирование технических Н. как своеобразного опосредующего слоя знания между естествознанием и производством, а затем произошло становление социальных Н. 
Каждый из этих этапов имел свои социокультурные предпосылки. Первый образец математической теории (евклидова геометрия) возник в контексте антич. культуры с присущими ей ценностями публичной дискуссии, демонстрации, доказательства и обоснования как условиями получения истины. 
Естествознание, основанное на соединении математического описания природы с ее экспериментальным исследованием, формировалось в результате культурных сдвигов, происшедших в эпохи Ренессанса, Реформации и раннего Просвещения. 
Становление технических и социальных Н. было вызвано интенсивным индустриальным развитием, стимулируемым внедрением научных знаний в производство и возникновением потребностей научного управления социальными процессами. 
На каждом из этапов развития научное познание усложняло свою организацию. Во всех развитых Н. складываются свои уровни теоретического и эмпирического исследований со специфическими для них методами и формами знания (основной формой теоретического уровня выступает теория научная; эмпирического уровня — научный факт). 
К сер. 19 в. формируется дисциплинарная организация Н., возникает сложная система взаимосвязанных дисциплин. Каждая из Н. (математика, физика, химия, биология, технические и социальные Н.) имеет свою внутреннюю дифференциацию и свои основания — свойственную ей картину исследуемой реальности, специфику идеалов и норм исследования и характерные для нее философско-мировоззренческие идеи. 
Взаимодействие Н. формирует междисциплинарные исследования, удельный вес которых возрастает по мере развития Н. 
Каждый этап развития Н. сопровождался особым типом ее институализации, определяемой организацией исследований и способом воспроизводства субъекта научной деятельности. Как социальный ин-т Н. начала оформляться в 17—18 вв., когда в Европе возникли первые научные общества, академии и научные журналы. В 20 в. Н. превратилась в особый тип производства научных знаний, включающий многообразные объединения ученых, в т.ч. крупные исследовательские коллективы, целенаправленное финансирование и особую экспертизу исследовательских программ, их социальную поддержку, специфическую промышленно-техническую базу, обслуживающую научный поиск, сложное разделение труда и целенаправленную подготовку кадров. 
В процессе исторического развития Н. менялись ее функции в социальной жизни. В эпоху становления естествознания Н. отстаивала в борьбе с религией право участвовать в формировании мировоззрения. В 19 в. к мировоззренческой функции добавилась функция быть производительной силой. В первой пол. 20 в. Н. стала приобретать еще одну функцию — превращаться в социальную силу, внедряясь в различные сферы социальной жизни и регулируя различные виды человеческой деятельности. 
В современную эпоху, в связи с глобальными кризисами возникает проблема поиска новых мировоззренческих ориентаций человечества. В этой связи переосмысливаются и функции Н. Ее доминирующее положение в системе ценностей культуры во многом было вызвано ее техническим приложением. Сегодня важно органическое соединение ценностей научно-технологического мышления с теми социальными ценностями, которые представлены нравственностью, искусством, религиозным и филос. постижением мира. Такое соединение представляет собой новый тип научной рациональности. В развитии Н. начиная с 17 в. можно выделить три основных типа рациональности: классическую (17 — нач. 20 в.), неклассическую (пер. пол. 20 в.), п о с т н е к л а с -сическую (кон. 20 в.). 
Классическая Н. предполагала, что субъект дистанцирован от объекта и как бы со стороны познает мир; условием объективно истинного знания считалась элиминация из объяснения и описания всего, что относится к субъекту и средствам деятельности. Для неклассической рациональности характерна идея отнесенности объекта к средствам и операциям деятельности; экспликация этих средств и операций выступает условием получения истинного знания об объекте. Образец реализации такого подхода — квантово-релятивистская физика. Наконец, постнеклассическая рациональность учитывает соотнесенность знаний об объекте не только со средствами, но и с ценностно-целевыми структурами деятельности, предполагая экспликацию внутринаучных ценностей и их соотнесение с социальными целями и ценностями. 
Появление каждого нового типа рациональности не устраняет предыдущий тип, но ограничивает поле его действия. Каждый из этих типов расширяет поле исследуемых объектов. 
В современной, постнеклассической Н. все большее место занимают сложные, исторически развивающиеся системы, включающие человека. К ним относятся объекты современных биотехнологий, в первую очередь генной инженерии, медико-биологические объекты, крупные экосистемы и биосфера в целом, человеко-машинные системы, включая системы искусственного интеллекта, социальные объекты, и т.д. 
В широком смысле сюда можно отнести любые сложные синергетические системы, взаимодействие с которыми превращает само человеческое действие в компонент системы. Методология исследования таких объектов сближает естественно-научное и гуманитарное познание, составляя основу для их глубокой интеграции.
\end{document}