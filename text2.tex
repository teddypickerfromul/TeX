\documentclass[a4paper,11pt]{article}
\usepackage[warn]{mathtext} 
\usepackage[russian]{babel} 
\usepackage[utf8]{inputenc}
\usepackage{graphicx}
\usepackage[top=1cm,bottom=1.5cm,left=1cm,right=1cm]{geometry}
\usepackage{multirow}
\usepackage{mcaption}

\begin{document}
		\begin{flushleft}
		\textit{\textbf{Задание :}}
		\\
		\medskip
		\hangindent=1.5cm \hangafter=0 \noindent
		Применить прямое преобразование \textbf{DFT} к данному сигналу и по результатам построить график для значений ${X_k}$ от ${k}$.
		\end{flushleft}
		\begin{flushleft}
		\textit{\textbf{Задание c учетом варианта 7 :}}
		\\
		\medskip
		\hangindent=1.5cm \hangafter=0 \noindent
		Дан сигнал следующего вида :		
	\end{flushleft}
			\begin{center}
					\begin{tabular}{|c|c|c|c|c|c|c|c|c|c|}
\hline 29.7823 & 31.1328 & 30.2371 & 30.3639 & 31.6048 & 30.68016 & 30.2778 & 33.0210 & 30.0855 & 33.2741 \\ 
\hline 30.7759 & 31.5608 & 30.5677 & 29.7243 & 31.4177 & 32.0702 & 31.0202 & 33.0210 & 30.2715 & 30.6582 \\ 
\hline 34.3480 & 32.2952 & 31.0319 & 31.5736 & 31.7112 & 30.4105 & 31.5504 & 31.0539 & 31.1867 & 31.7484 \\ 
\hline 31.8624 & 30.6155 & 31.3820 & 30.6208 & 31.5235 & 29.9919 & 29.9241 & 31.5702 & 30.7358 & 30.6827 \\ 
\hline 
\end{tabular}
		\end{center}
		\begin{flushleft}
		\medskip
		\hangindent=1.5cm \hangafter=0 \noindent
		\textit{Примечание :}
		\\
		\medskip В данной таблице указаны округленные значения, более точные значения содержатся в файле задания к данной лабораторной работе (т.е. ${33.0210456848 \approx 33.0210}$ ).В расчетах использовались точные значения из файла задания. 	
		\end{flushleft}
		\begin{flushleft}
		\textit{\textbf{Решение :}}
		\\
		\medskip
		\hangindent=1.5cm \hangafter=0 \noindent
		Прямое преобразование \textbf{DFT} заданного сигнала будем производить используя следующую формулу :
		\begin{equation}
			X_k = \sum_{n=0}^{N-1} x_n \cdot \exp^{\frac{-2 \cdot \pi }{N} \cdot k \cdot n}
		\end{equation}
		Представив экспоненту суммой тригонометрических функций, получим :
		\begin{equation}
			X_k = \sum_{n=0}^{N-1} x_n \cdot (\cos({\frac{2 \cdot \pi }{N} \cdot k \cdot n}) + i \cdot \sin({\frac{2 \cdot \pi }{N} \cdot k \cdot n}))			
		\end{equation}
		где ${N}$ — количество значений сигнала, измеренных за период, а также количество компонент разложения;
		\\
		${x_n}$ — измеренные значения сигнала (в дискретных временных точках с номерами , ${n = 0...N-1}$ которые являются входными данными для прямого преобразования и выходными для обратного;
		\\
		${X_k}$ — $N$ комплексных амплитуд синусоидальных сигналов, слагающих исходный сигнал; являются выходными данными для прямого преобразования и входными для обратного; поскольку амплитуды комплексные, то по ним можно вычислить одновременно и амплитуду, и фазу.
		\\
		\medskip
		Используя это можно найти вещественную амплитуду $k$-го синусоидального сигнала;
		\begin{equation}
			\frac{|X_k|}{N}
		\end{equation}
		Фаза $k$-го синусоидального сигнала (аргумент комплексного числа) тогда имеет вид :
		\begin{equation}
			\arg(X_k)
		\end{equation}
		$k$ — частота $k$-го сигнала, равная ${\frac{k}{T}}$, где $T$ — период времени, в течение которого брались входные данные.
		\\		
		\bigskip
		Для расчетов использовалась среда \textbf{Scilab}, для которой была написана программа, получающая на вход данный сигнал в виде вектора действительных чисел, и вычисляющая для всех заданных $k$ нужные для расчетов выражения под суммой в формуле \textit{(2)} и печатающая эти значения на экран, а также непосредственно значения комплексную амплитуду для заданной частоты $k$.\\
		Также вычисляющая значения вещественных амплитуд и фаз сигнала.
		\end{flushleft}
		\newpage
		\begin{center}
			\includegraphics[scale=0.7]{sc1.png}
		\end{center}
		\begin{flushleft}
		\medskip
		\hangindent=1.5cm \hangafter=0 \noindent
		Исходный код программы представлен ниже :
		\end{flushleft}	
		\begin{center}
			\includegraphics[scale=0.78]{code.png}			
		\end{center}
		\newpage
		\begin{flushleft}
		\textit{\textbf{Результаты вычислений :}}
		\\
		\medskip
		\hangindent=1.5cm \hangafter=0 \noindent
		
\textbf{Полагаем k=0 :\\ }

 ---> Амплитула (комплексная) для k = 0 :  1246.9772\\ 
 ==> Действительная чаcть амплитуды  для k = 0 : 1246.9772\\ 
 ==> Мнимая часть амплитуды  для k = 0 : 0\\ 
 ==> Амплитуда (вещественная) = 31.174431\\ 
 ==> Фаза = 0\\ 
\textbf{Полагаем k=1 :\\} 

 ==> Амплитула (комплексная) для k = 1 : -9.6948511-i*0.5528856 \\ 
 ==> Действительная чаcть амплитуды  для k = 1 : -9.6948511\\ 
 ==> Мнимая часть амплитуды  для k = 1 : -0.5528856\\ 
 ==> Амплитуда (вещественная) = 0.2427651\\ 
 ==> Фаза = -3.0846256\\ 

\textbf{Полагаем k=2 :\\} 

 ==> Амплитула (комплексная) для k = 2 : -1.5273665+i*2.3344515\\ 
 ==> Действительная чаcть амплитуды  для k = 2 : -1.5273665\\ 
 ==> Мнимая часть амплитуды  для k = 2 : 2.3344515\\ 
 ==> Амплитуда (вещественная) = 0.0697429\\ 
 ==> Фаза = 2.1501689\\ 
\textbf{Полагаем k=3 :\\ }
 ==> Амплитула (комплексная) для k = 3 : -5.6918595-i*2.5481615 \\ 
 ==> Действительная чаcть амплитуды  для k = 3 : -5.6918595 \\ 
 ==> Мнимая часть амплитуды  для k = 3 : -2.5481615\\ 
 ==> Амплитуда (вещественная) = 0.1559054\\ 
 ==> Фаза = -2.7206653\\  
\textbf{Полагаем k=4 :\\ }
 ==> Амплитула (комплексная) для k = 4 :  5.4666828+i*0.0111185\\ 
 ==> Действительная чаcть амплитуды  для k = 4 : 5.4666828\\ 
 ==> Мнимая часть амплитуды  для k = 4 : 0.0111185\\ 
 ==> Амплитуда (вещественная) = 0.1366674\\ 
 ==> Фаза = 0.0020339\\ 
\textbf{Полагаем k=5 :\\} 
 ==> Амплитула (комплексная) для k = 5 :  2.4373452-i*1.2146069\\ 
 ==> Действительная чаcть амплитуды  для k = 5 : 2.4373452\\ 
 ==> Мнимая часть амплитуды  для k = 5 : -1.2146069\\ 
 ==> Амплитуда (вещественная) = 0.0680805\\ 
 ==> Фаза = -0.4623123\\ 

\textbf{Полагаем k=6 :\\ }
 
 ==> Амплитула (комплексная) для k = 6 : -3.9720695+i*1.8152452\\ 
 ==> Действительная чаcть амплитуды для k = 6 : -3.9720695\\ 
 ==> Мнимая часть амплитуды  для k = 6 : 1.8152452\\ 
 ==> Амплитуда (вещественная) = 0.1091800\\ 
 ==> Фаза = 2.7129308\\ 
\textbf{Полагаем k=7 :\\} 
 ==> Амплитула (комплексная) для k = 7 :  2.5759636-i*6.3633509\\ 
 ==> Действительная чаcть амплитуды  для k = 7 : 2.5759636\\ 
 ==> Мнимая часть амплитуды  для k = 7 : -6.3633509\\ 
 ==> Амплитуда (вещественная) = 0.1716243\\ 
 ==> Фаза = -1.1861482\\ 

\textbf{Полагаем k=8 :\\ }

 ==> Амплитула (комплексная) для k = 8 :  1.0200864+i*3.2643594\\ 
 ==> Действительная чаcть амплитуды  для k = 8 : 1.0200864\\ 
 ==> Мнимая часть амплитуды  для k = 8 : 3.2643594\\ 
 ==> Амплитуда (вещественная) = 0.0855008\\ 
 ==> Фаза = 1.2679187\\ 

\textbf{Полагаем k=9 :\\ }

 ==> Амплитула (комплексная) для k = 9 : -0.2813068-i*7.0748464\\ 
 ==> Действительная чаcть амплитуды  для k = 9 : -0.2813068\\ 
 ==> Мнимая часть амплитуды  для k = 9 : -7.0748464\\ 
 ==> Амплитуда (вещественная) = 0.1770109\\ 
 ==> Фаза = -1.6105369\\ 

\textbf{Полагаем k=10 :\\ }

 ==> Амплитула (комплексная) для k = 10 : -2.4951382+i*1.9285107\\ 
 ==> Действительная чаcть амплитуды  для k = 10 : -2.4951382\\ 
 ==> Мнимая часть амплитуды  для k = 10 : 1.9285107\\ 
 ==> Амплитуда (вещественная) = 0.0788387\\ 
 ==> Фаза = 2.4835913\\ 

\textbf{Полагаем k=11 :\\} 

 ==> Амплитула (комплексная) для k = 11 : -1.5015807-i*2.4878853\\ 
 ==> Действительная чаcть амплитуды  для k = 11 : -1.5015807\\ 
 ==> Мнимая часть амплитуды  для k = 11 : -2.4878853\\ 
 ==> Амплитуда (вещественная) = 0.0726478\\ 
 ==> Фаза = -2.1138272\\ 
\textbf{Полагаем k=12 :\\} 

 ==> Амплитула (комплексная) для k = 12 : -7.3923629+i*4.749517\\ 
 ==> Действительная чаcть амплитуды  для k = 12 : -7.3923629\\ 
 ==> Мнимая часть амплитуды  для k = 12 : 4.749517\\ 
 ==> Амплитуда (вещественная) = 0.2196659\\ 
 ==> Фаза = 2.5705152\\ 

\textbf{Полагаем k=13 :\\} 
 ==> Амплитула (комплексная) для k = 13 :  2.6828477-i*4.5304831\\ 
 ==> Действительная чаcть амплитуды  для k = 13 : 2.6828477\\ 
 ==> Мнимая часть амплитуды  для k = 13 : -4.5304831\\ 
 ==> Амплитуда (вещественная) = 0.1316315\\ 
 ==> Фаза = -1.0361489\\ 

\textbf{Полагаем k=14 :\\ }
 ==> Амплитула (комплексная) для k = 14 : -8.390877+i*0.4035128\\ 
 ==> Действительная чаcть амплитуды  для k = 14 : -8.390877\\ 
 ==> Мнимая часть амплитуды  для k = 14 : 0.4035128\\ 
 ==> Амплитуда (вещественная) = 0.2100143\\ 
 ==> Фаза = 3.0935402\\ 

\textbf{Полагаем k=15 :\\} 

 ==> Амплитула (комплексная) для k = 15 :  6.2313789-i*5.3153493\\ 
 ==> Действительная чаcть амплитуды  для k = 15 : 6.2313789\\ 
 ==> Мнимая часть амплитуды  для k = 15 : -5.3153493\\ 
 ==> Амплитуда (вещественная) = 0.2047606\\ 
 ==> Фаза = -0.7062316\\ 

\textbf{Полагаем k=16 :\\} 

 ==> Амплитула (комплексная) для k = 16 :  7.0436915+i*6.1147644\\ 
 ==> Действительная чаcть амплитуды  для k = 16 : 7.0436915\\ 
 ==> Мнимая часть амплитуды  для k = 16 : 6.1147644\\ 
 ==> Амплитуда (вещественная) = 0.2331897\\ 
 ==> Фаза = 0.7149196\\ 

\textbf{Полагаем k=17 :\\} 
 ==> Амплитула (комплексная) для k = 17 :  1.2891756+i*2.8276264\\ 
 ==> Действительная чаcть амплитуды  для k = 17 : 1.2891756\\ 
 ==> Мнимая часть амплитуды  для k = 17 : 2.8276264\\ 
 ==> Амплитуда (вещественная) = 0.0776911\\ 
 ==> Фаза = 1.143029\\ 
\textbf{Полагаем k=18 :\\ }

 ==> Амплитула (комплексная) для k = 18 : -2.0224086-i*0.1704630\\ 
 ==> Действительная чаcть амплитуды  для k = 18 : -2.0224086\\ 
 ==> Мнимая часть амплитуды  для k = 18 : -0.1704630\\ 
 ==> Амплитуда (вещественная) = 0.0507395\\ 
 ==> Фаза = -3.0575043\\ 
\textbf{Полагаем k=19 :\\} 

 ==> Амплитула (комплексная) для k = 19 :  2.1983819-i*9.087769\\ 
 ==> Действительная чаcть амплитуды  для k = 19 : 2.1983819\\ 
 ==> Мнимая часть амплитуды  для k = 19 : -9.087769\\ 
 ==> Амплитуда (вещественная) = 0.2337472\\ 
 ==> Фаза = -1.3334503\\ 
\textbf{Полагаем k=20 :\\} 

 ==> Амплитула (комплексная) для k = 20 :  4.3824043-i*2.153D-13\\ 
 ==> Действительная чаcть амплитуды  для k = 20 : 4.3824043\\ 
 ==> Мнимая часть амплитуды  для k = 20 : -2.153D-13\\ 
 ==> Амплитуда (вещественная) = 0.1095601\\ 
 ==> Фаза = -4.914D-14\\ 

\textbf{Полагаем k=21 :\\ }

 ==> Амплитула (комплексная) для k = 21 :  2.1983819+i*9.087769\\ 
 ==> Действительная чаcть амплитуды  для k = 21 : 2.1983819\\ 
 ==> Мнимая часть амплитуды  для k = 21 : 9.087769\\ 
 ==> Амплитуда (вещественная) = 0.2337472\\ 
 ==> Фаза = 1.3334503\\ 

\textbf{Полагаем k=22 :\\} 

 ==> Амплитула (комплексная) для k = 22 : -2.0224086+i*0.1704630\\ 
 ==> Действительная чаcть амплитуды  для k = 22 : -2.0224086\\ 
 ==> Мнимая часть амплитуды  для k = 22 : 0.1704630\\ 
 ==> Амплитуда (вещественная) = 0.0507395\\ 
 ==> Фаза = 3.0575043\\ 

\textbf{Полагаем k=23 :\\} 

 ==> Амплитула (комплексная) для k = 23 :  1.2891756-i*2.8276264\\ 
 ==> Действительная чаcть амплитуды  для k = 23 : 1.2891756\\ 
 ==> Мнимая часть амплитуды  для k = 23 : -2.8276264\\ 
 ==> Амплитуда (вещественная) = 0.0776911\\ 
 ==> Фаза = -1.143029\\ 

\textbf{Полагаем k=24 :\\ }

 ==> Амплитула (комплексная) для k = 24 :  7.0436915-i*6.1147644\\ 
 ==> Действительная чаcть амплитуды  для k = 24 : 7.0436915\\ 
 ==> Мнимая часть амплитуды  для k = 24 : -6.1147644\\ 
 ==> Амплитуда (вещественная) = 0.2331897\\ 
 ==> Фаза = -0.7149196\\ 

\textbf{Полагаем k=25 :\\} 

 ==> Амплитула (комплексная) для k = 25 :  6.2313789+i*5.3153493\\ 
 ==> Действительная чаcть амплитуды  для k = 25 : 6.2313789\\ 
 ==> Мнимая часть амплитуды  для k = 25 : 5.3153493\\ 
 ==> Амплитуда (вещественная) = 0.2047606\\ 
 ==> Фаза = 0.7062316\\ 

\textbf{Полагаем k=26 :\\} 

 ==> Амплитула (комплексная) для k = 26 : -8.390877-i*0.4035128\\ 
 ==> Действительная чаcть амплитуды  для k = 26 : -8.390877\\ 
 ==> Мнимая часть амплитуды  для k = 26 : -0.4035128\\ 
 ==> Амплитуда (вещественная) = 0.2100143\\ 
 ==> Фаза = -3.0935402\\ 

\textbf{Полагаем k=27 :\\} 

 ==> Амплитула (комплексная) для k = 27 :  2.6828477+i*4.5304831\\ 
 ==> Действительная чаcть амплитуды  для k = 27 : 2.6828477\\ 
 ==> Мнимая часть амплитуды  для k = 27 : 4.5304831\\ 
 ==> Амплитуда (вещественная) = 0.1316315\\ 
 ==> Фаза = 1.0361489\\ 

\textbf{Полагаем k=28 :\\} 

 ==> Амплитула (комплексная) для k = 28 : -7.3923629-i*4.749517\\ 
 ==> Действительная чаcть амплитуды  для k = 28 : -7.3923629\\ 
 ==> Мнимая часть амплитуды  для k = 28 : -4.749517\\ 
 ==> Амплитуда (вещественная) = 0.2196659\\ 
 ==> Фаза = -2.5705152\\ 

\textbf{Полагаем k=29 :\\} 

 ==> Амплитула (комплексная) для k = 29 : -1.5015807+i*2.4878853\\ 
 ==> Действительная чаcть амплитуды  для k = 29 : -1.5015807\\ 
 ==> Мнимая часть амплитуды  для k = 29 : 2.4878853\\ 
 ==> Амплитуда (вещественная) = 0.0726478\\ 
 ==> Фаза = 2.1138272\\ 
\textbf{Полагаем k=30 :\\} 

 ==> Амплитула (комплексная) для k = 30 : -2.4951382-i*1.9285107\\ 
 ==> Действительная чаcть амплитуды  для k = 30 : -2.4951382\\ 
 ==> Мнимая часть амплитуды  для k = 30 : -1.9285107\\ 
 ==> Амплитуда (вещественная) = 0.0788387\\ 
 ==> Фаза = -2.4835913\\ 

\textbf{Полагаем k=31 :\\} 

 ==> Амплитула (комплексная) для k = 31 : -0.2813068+i*7.0748464\\ 
 ==> Действительная чаcть амплитуды  для k = 31 : -0.2813068\\ 
 ==> Мнимая часть амплитуды  для k = 31 : 7.0748464\\ 
 ==> Амплитуда (вещественная) = 0.1770109\\ 
 ==> Фаза = 1.6105369\\ 

\textbf{Полагаем k=32 :\\} 

 ==> Амплитула (комплексная) для k = 32 :  1.0200864-i*3.2643594\\ 
 ==> Действительная чаcть амплитуды  для k = 32 : 1.0200864\\ 
 ==> Мнимая часть амплитуды  для k = 32 : -3.2643594\\ 
 ==> Амплитуда (вещественная) = 0.0855008\\ 
 ==> Фаза = -1.2679187\\ 

\textbf{Полагаем k=33 :\\} 

 ==> Амплитула (комплексная) для k = 33 :  2.5759636+i*6.3633509\\ 
 ==> Действительная чаcть амплитуды  для k = 33 : 2.5759636\\ 
 ==> Мнимая часть амплитуды  для k = 33 : 6.3633509\\ 
 ==> Амплитуда (вещественная) = 0.1716243\\ 
 ==> Фаза = 1.1861482\\ 

\textbf{Полагаем k=34 :\\} 
 ==> Амплитула (комплексная) для k = 34 : -3.9720695-i*1.8152452\\ 
 ==> Действительная чаcть амплитуды  для k = 34 : -3.9720695\\ 
 ==> Мнимая часть амплитуды  для k = 34 : -1.8152452\\ 
 ==> Амплитуда (вещественная) = 0.1091800\\ 
 ==> Фаза = -2.7129308\\ 

\textbf{Полагаем k=35 :\\} 

 ==> Амплитула (комплексная) для k = 35 :  2.4373452+i*1.2146069\\ 
 ==> Действительная чаcть амплитуды  для k = 35 : 2.4373452\\ 
 ==> Мнимая часть амплитуды  для k = 35 : 1.2146069\\ 
 ==> Амплитуда (вещественная) = 0.0680805\\ 
 ==> Фаза = 0.4623123\\ 

\textbf{Полагаем k=36 :\\} 

 ==> Амплитула (комплексная) для k = 36 :  5.4666828-i*0.0111185\\ 
 ==> Действительная чаcть амплитуды  для k = 36 : 5.4666828\\ 
 ==> Мнимая часть амплитуды  для k = 36 : -0.0111185\\ 
 ==> Амплитуда (вещественная) = 0.1366674\\ 
 ==> Фаза = -0.0020339\\ 

\textbf{Полагаем k=37 :\\} 
 ==> Амплитула (комплексная) для k = 37 : -5.6918595+i*2.5481615\\ 
 ==> Действительная чаcть амплитуды  для k = 37 : -5.6918595\\ 
 ==> Мнимая часть амплитуды  для k = 37 : 2.5481615\\ 
 ==> Амплитуда (вещественная) = 0.1559054\\ 
 ==> Фаза = 2.7206653\\ 

\textbf{Полагаем k=38 :\\} 
 ==> Амплитула (комплексная) для k = 38 : -1.5273665-i*2.3344515\\ 
 ==> Действительная чаcть амплитуды  для k = 38 : -1.5273665\\ 
 ==> Мнимая часть амплитуды  для k = 38 : -2.3344515\\ 
 ==> Амплитуда (вещественная) = 0.0697429\\ 
 ==> Фаза = -2.1501689\\ 

\textbf{Полагаем k=39 :\\} 

 ==> Амплитула (комплексная) для k = 39 : -9.6948511+i*0.5528856\\ 
 ==> Действительная чаcть амплитуды  для k = 39 : -9.6948511\\ 
 ==> Мнимая часть амплитуды  для k = 39 : 0.5528856\\ 
 ==> Амплитуда (вещественная) = 0.2427651\\ 
 ==> Фаза = 3.0846256\\ 

\textbf{Полагаем k=40 :\\} 
 ==> Амплитула (комплексная) для k = 40 :  1246.9772-i*6.960D-12\\ 
 ==> Действительная чаcть амплитуды  для k = 40 : 1246.9772\\ 
 ==> Мнимая часть амплитуды  для k = 40 : -6.960D-12\\ 
 ==> Амплитуда (вещественная) = 31.174431\\ 
 ==> Фаза = -5.582D-15\\ 

 		
		\end{flushleft}
		\newpage
		\begin{flushleft}
		\textit{\textbf{Графики зависимости ${X_k}$ :}}
		\\
		\hangindent=1.5cm \hangafter=0 \noindent
		Амплитуда :
		\begin{center}
			\includegraphics[scale=0.71]{ampl_final.png}
		\end{center}
		Как видно из предыдущего графика, разность между амплитудными значениями и знач. между ними велика, исключим точки ${k=0}$ и ${k=40}$ :
		\begin{center}
			\includegraphics[scale=0.75]{ampl2_final.png}	
		\end{center}
		Фаза :
		\begin{center}
			\includegraphics[scale=0.75]{faze_final.png}
		\end{center} 
		\end{flushleft}				
\end{document}
