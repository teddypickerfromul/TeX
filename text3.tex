\documentclass[a4paper,12pt]{article}
\usepackage[russian]{babel} 
\usepackage[utf8x]{inputenc}
\usepackage{graphicx}
\usepackage[top=1cm,bottom=1.5cm,left=1cm,right=1cm]{geometry}
\usepackage{listings} 
\begin{document}
 	\begin{center}
 		\begin{Large}
 			Распознавание образов
 		\end{Large}
 	\end{center}
 	
 	\begin{flushleft}
 		\textit{\textbf{Цель работы :}}
 		\\
 		\medskip
 		\hangindent=1.5cm \hangafter=0 \noindent
 		Получение практических навыков работы с простейшими алгоритмами 		распознавания объектов с качественными характеристиками.
 	\end{flushleft}
 		
	\begin{flushleft}
		\textit{\textbf{Задание :}}
		\\
		\medskip
		\hangindent=1.5cm \hangafter=0 \noindent
		На любом языке программирования реализовать программу, определяющую  степень сходства образцов, поданных на вход. Использовать при этом все формулы сходства и проанализировать полученные результаты.Сформулировать признаки, по которым будет осуществляться сравнение (не менее 5).
	\end{flushleft}
	
	\begin{flushleft}
		\textit{\textbf{Задание c учетом варианта :}}
		\\
		\begin{center}
				\begin{tabular}{|c|c|}
					\hline \textbf{Номер варианта} & \textbf{Эталоны} \\ 
					\hline 6 & локальная сеть, Интернет, интранет \\ 
					\hline 
				\end{tabular} 	
		\end{center}	
	\end{flushleft}
	\begin{flushleft}
		\textit{\textbf{Общее описание алгоритма :}}
		\\
		\medskip
		\hangindent=1.5cm \hangafter=0 \noindent
		В большинстве случаев образы и отдельные изображения характеризуются с помощью количественных характеристик: геометрических размеров, веса, площади, объема и т. д. В этих случаях количественные изменения характеристик конкретного изображения обычно не сразу ведут к изменению образа, к которому относится распознаваемое изображение. Только достигнув определенных для каждого образа границ, количественные изменения вызывают качественный скачок - переход к другому образу. Образы и конкретные изображения могут характеризоваться не только количественными, но и качественными характеристиками (свойствами, признаками, атрибутами). Эти признаки не могут быть описаны (или обычно не описываются) количественно, например, цвет, вкус, ощущение, запах. Образы либо обладают какими-то качественными характеристиками, либо не обладают.
		\\
		\medskip
		Существуют различные подходы к распознаванию изображений с качественными характеристиками. В данной лабораторной работе мы используем один из них, основанный на двоичном кодировании наличия или отсутствия какого-либо качественного признака. В рассматриваемом подходе конкретное изображение ${X_k}$ некоторого образа с качественными характеристиками представляется в виде двоичного вектора
		\\
		\medskip
		Для указанных в данном варианте эталонов выделим 5 качественных признаков таким образом, что бы соответсвующие данным эталонам векторы значений качественных признаков не совпадали :
		\begin{list}{\labelitemi}{\leftmargin=3cm}
				\item Частность
				\item Глобальность
				\item Высокоуровневость 
				\item Наличие администратора
				\item Наличие владельца (хозяин)
		\end{list}
		\medskip
		Для кодирования наличия этих признаков у эталонов используем 		двоичные векторы характеристик(в программной реализации это одномерные массивы элементов типа int с значениями 1("наличие") или 0("отсутсвие")) 
		\\
		\medskip
		В результате имеем следующую таблицу :
		\\
		\medskip
\begin{center}
		\begin{tabular}{|c|c|c|c|c|c|}
			\hline \textbf{Эталон} & \textbf{Частность} & 			\textbf{Глобальность} & \textbf{Высокоур.} & \textbf{Администратор} & 		\textbf{Владелец} \\ 
			\hline Интранет & Да & Нет & Да & Да & Да \\ 
			\hline Локальная сеть & Да & Нет & Нет & Да & Нет \\ 
			\hline Интернет & Нет & Да & Да & Нет & Нет \\ 
			\hline 
		\end{tabular}
		\end{center}
		\medskip
		Реализация которой тривиальна и представлена на скриншоте : 
		\begin{center}
			\includegraphics[]{test.png}
		\end{center} 
		Имея готовые характеристические векторы для заданных эталонов, мы должны заполнить аналогичным образом характеристический вектор для заданной пользователем сущности.\\Заполнив соотвествующим образом этот вектор, мы можем произвести более тонкую классификацию объектов с качественными признаками получается при введении для каждой пары объектов
		${X_j}$, ${X_n}$, для которых введено двоичное кодирование качественных признаков, переменных : ${a,b,g,h}$, формулы для нахождения которых представлены ниже :
				 
		\begin{equation}
				a = \sum_{k=1}^n x_{jk}\cdot x_{ik}	
		\end{equation}
		\begin{equation}
				b = \sum_{k=1}^n (1 - x_{jk})\cdot(1 - x_{ik})
		\end{equation}
		\begin{equation}
				g = \sum_{k=1}^n x_{jk}\cdot (1 - x_{ik})
		\end{equation}
		\begin{equation}
				h = \sum_{k=1}^n (1 - x_{jk})\cdot x_{ik} 
		\end{equation}
		\medskip
 	Пример реализации вычисления одного из таких коэффициентов представлен нв скриншоте :
 		\begin{center}
			\includegraphics[]{test2.png}
		\end{center}
		Из анализа переменных ${a,b,g,h}$ следует, что, чем больше сходство между объектами ${X_j}$, и ${X_i}$, тем больше должна быть переменная а, т.е. мера близости объектов или функция сходства должна быть возрастающей функцией от ${a}$, функция сходства должна быть симметричной относительно переменных ${g}$ и ${h}$. Относительно переменной ${b}$ однозначный вывод сделать не удается, поскольку, с одной стороны, отсутствие одинаковых признаков у объектов может свидетельствовать об их сходстве, однако, с другой стороны, если у объектов общим является только отсутствие одинаковых признаков, то они не могут относиться к одному классу.
		\\
		\medskip
		Полученные значения переменных ${a,b,g,h}$ используются в 7 наиболее часто встречаемых функциях сходства :
			\begin{list}{\labelitemi}{\leftmargin=3cm}
					\item \textit{Функция сходства Рассела и Рао}
					\item \textit{Функция сходства Жокара и Нидмена}
					\item \textit{Функция сходства Дайса} 
					\item \textit{Функция сходства Сокаля и Снифа}
					\item \textit{Функция сходства Сокаля и Мишнера}
					\item \textit{Функция сходства Кульжинского}
					\item \textit{Функция сходства Юла}
			\end{list}
		Пользуясь вычисленными коэффициентами и используя функции сходства можно 		обозначить общий ход алгоритма программы :	
			\begin{list}{\labelitemi}{\leftmargin=3cm}
				\item Для каждого характеристического вектора заданных эталонов :
					\begin{list}{\labelitemi}{\leftmargin=1.5cm}		
						\item Вычисляем переменные ${a,b,g,h}$
						\item Подставляем полученные переменные в каждую из семи функций сходства
						\item Вычисляем максимум из полученных значений функций сходства и сохраняем его в какой-нибудь структуре данных (обрабатывая особые случаи) 
					\end{list}			
				\item Находим максимум из максимумов полученных функций сходства для кажой итерации сравнения с каждым эталоном и получаем эталон, на который наиболее похожа сущность, определенная пользователем.
			\end{list}	 
	\end{flushleft}
	\begin{flushleft}
 		\textit{\textbf{Анализ функций сходства :}}
 		\\
 			\hangindent=1.5cm \hangafter=0 \noindent
			\begin{enumerate}
				\item \textbf{Функция сходства Рассела и Рао}
				\\ Аналитический вид :
					\begin{equation}
						S(x_i,x_j) = \frac{a}{a+b+g+h} = \frac{a}{n}
					\end{equation}
					Пределы функции :
					\\
					\medskip
					\begin{center}
						${\lim_{a\to \infty} \frac{a}{a+b+g+h}=1}$
						\\					
						\medskip						
						${\lim_{b\to \infty} \frac{a}{a+b+g+h}=0}$
						\\
						\medskip						
						${\lim_{g\to \infty} \frac{a}{a+b+g+h}=0}$
						\\
						\medskip						
						${\lim_{h\to \infty} \frac{a}{a+b+g+h}=0}$					
					\end{center}
					Поскольку переменные ${a,b,g,h}$ не могут одновременно быть 	равными нулю, то функция сходства Рассела и Рао не возвращает деления на 0.Функция возвращает 0 если ${a = 0}$.Если ${a=b=g=h=1}$ то функция вернет свое максимальное значение, равное 1.При использовании функции сходства Рассела и Рао степень распознания будет тем выше, чем больше одинаковых признаков у сравниваемого объекта и объекта-образца.
			\item \textbf{Функция сходства Жокара и Нидмена}
				\\ Аналитический вид :
					\begin{equation}
						S(x_i,x_j) = \frac{a}{n-b} 
					\end{equation}
					Пределы функции :
					\begin{center}
						${\lim_{a\to \infty} \frac{a}{a+g+h}=1}$
						\\					
						\medskip						
						${\lim_{g\to \infty} \frac{a}{a+g+h}=0}$
						\\
						\medskip						
						${\lim_{h\to \infty} \frac{a}{a+g+h}=0}$
						\\
						\medskip						
					\end{center}					
					В данном случае функция сходства Жокара и Нидмена имеет точки 	разрыва, то есть случаи, когда в знаменателе получается 0  - когда векторы 	сравниваемого объекта и объекта-образца представляют собой нулевые вектора.
					В такой ситуации мы получим неопределенность ${\frac{0}{0}}$, что дает нам 	право предполагать, что, вообще говоря, в этом случае функция сходства может принимать любое значение, т.к. о природе сравниваемого объекта и о природе объекта-образца ничего не известно, потому что они оба не обладают ни одним признаком, по которым производится сравнение.
			\item \textbf{Функция сходства Дайса}
				\\ Аналитический вид :
					\begin{equation}
						S(x_i,x_j) = \frac{a}{2\cdot a+g+h} 
					\end{equation}
					Пределы функции :
					\begin{center}
						${\lim_{a\to \infty} \frac{a}{2\cdot a+g+h}=0.5}$
						\\					
						\medskip						
						${\lim_{g\to \infty} \frac{a}{2\cdot a+g+h}=0}$
						\\
						\medskip						
						${\lim_{h\to \infty} \frac{a}{2\cdot a+g+h}=0}$
						\\
						\medskip						
					\end{center}
					Аналогично функции Жокара-Нидмена, но с максимумом 0.5.
			\item \textbf{Функция сходства Сокаля и Снифа}
				\\ Аналитический вид :
					\begin{equation}
						S(x_i,x_j) = \frac{a}{a + 2\cdot(g+h)} 
					\end{equation}
					Пределы функции :
					\begin{center}
						${\lim_{a\to \infty} \frac{a}{a + 2\cdot(g+h)}=1}$
						\\					
						\medskip						
						${\lim_{g\to \infty} \frac{a}{a + 2\cdot(g+h)}=0}$
						\\
						\medskip						
						${\lim_{h\to \infty} \frac{a}{a + 2\cdot(g+h)}=0}$
						\\
						\medskip					
					\end{center}
					Аналогично функции Жокара-Нидмена, однако различие в наличии определенного признака у сравниваемого объекта и объекта-образца снижает значение функции сходства.\\
			\item \textbf{Функция сходства Сокаля-Мишнера}
				\\ Аналитический вид :
					\begin{equation}
						S(x_i,x_j) = \frac{a+b}{n} 
					\end{equation}
					Пределы функции :					 													\begin{center}
						${\lim_{a\to \infty} \frac{a+b}{n)}=1}$
						\\					
						\medskip						
						${\lim_{b\to \infty} \frac{a+b}{n}=1}$
						\\
						\medskip						
						${\lim_{g\to \infty} \frac{a+b}{n}=0}$
						\\
						\medskip
						${\lim_{h\to \infty} \frac{a+b}{n}=0}$
						\\
						\medskip											
					\end{center}
					Минимальное значение функции равно 0, а максимальное - 1.Всякое расхождение в наличии признаков ведет к уменьшению значения этой функции сходства.					\item \textbf{Функция сходства Кульжинского}
				\\ Аналитический вид :
					\begin{equation}
						S(x_i,x_j) = \frac{a}{g+h} 
					\end{equation}
					Пределы функции :
					\begin{center}
						${\lim_{g\to \infty} \frac{a}{g+h)}=0}$
						\\					
						\medskip						
						${\lim_{h\to \infty} \frac{a}{g+h}=0}$
						\\
						\medskip
					\end{center}
					Аналогично функции Жокара-Нидмена с добавлением еще одного критичного случая, когда имеется хотя бы один признак, который присутствует у сравниваемого объекта и объекта-образца, и нет признаков, которые есть у одного объекта, и нет у другого.
					В таком случае получаем неопределенность ${\frac{a}{0}}$, что дает значение функции, равное ${\infty}$, т.е. как бы сравниваемый объект и объект-образец идентичны.
			\item \textbf{Функция сходства Юла}
				\\ Аналитический вид :
					\begin{equation}
						S(x_i,x_j) = \frac{a\cdot b-g\cdot h}{a\cdot b+g\cdot h} 
					\end{equation}
					Пределы функции :																		\\
					\medskip
					\begin{center}
						${\lim_{a\to \infty} \frac{a\cdot b-g\cdot h}{a\cdot b+g\cdot h}=1}$
						\\					
						\medskip						
						${\lim_{b\to \infty} \frac{a\cdot b-g\cdot h}{a\cdot b+g\cdot h}=1}$
						\\
						\medskip						
						${\lim_{g\to \infty} \frac{a\cdot b-g\cdot h}{a\cdot b+g\cdot h}=-1}$
						\\
						\medskip						
						${\lim_{h\to \infty} \frac{a\cdot b-g\cdot h}{a\cdot b+g\cdot h}=-1}$					
					\end{center}
Значения функции лежат в отрезке ${[-1;1]}$. При этом существуют точки разрыва, дающие неопределенность вида ${\frac{0}{0}}$. Это говорит о том, что должен быть хотя бы 1 признак, который есть у обоих объектов, 1 признак, которого нет у обоих объектов (${ab \ne 0}$), или 1 признак, который есть у одного объекта и которого нет у другого (и наоборот) (${gh \ne 0}$).
\\
\medskip
В противном случае функцию сходства Юла применять нельзя, т.к. она может принимать любое значение.\\
\bigskip
Из анализа вышеперечисленных семи функций сходства ясно, что осбое внимание при программной реализации стоит уделять функции сходства Кульжинского и фунции сходства Юла.В данном варианте при заданных характеристических векторах эталонов только функция сходства Кульжинского возвращала неопределенность вида ${\frac{n}{0}}$ в случаях полного совпадения эталона и образца, а также если характеристический вектор образца содержал одни нули и единицы. 	
			\end{enumerate}				 
 	\end{flushleft}
 	\begin{flushleft}
 		\textit{\textbf{Программная реализация :}}
			\\
			\medskip
			\hangindent=1.5cm \hangafter=0 \noindent
			Данная лабораторная работа была написана на языке программирования C++ (c некоторыми расширениями).Для реализации GUI приложения был выбран кроссплатформенный фреймворк для разработки desktop-приложений QT(точнее - его модули  QTCore и QTGUI).Интерфейс приложения состоит имеет вкладочный вид и состоит из 2х вкладок : ввод данных и вкладка результатов.\\
			\medskip
			Вкладка "Ввод данных" имеет следующий вид :
			\begin{center}
				\includegraphics[scale=0.6]{pr_1.png}
			\end{center}
			Эта вкладка содержит флажки, соответсвующие наличие определенных признаков у исследуемого обьекта;вжджет-таблицу, где отображаются соответсвующие признаки у исследуемого обьекта и предопределенных эталонов в виде значений типа Да/Нет.Переключение флажков ведет к изменению значений соотвествующих признаков в таблице.Сравнение эталона с образцом происходит при нажатии соотвествующей кнопки.
			\newpage
			Вкладка "Результаты" имеет следующий вид :
			\begin{center}
				\includegraphics[scale=0.6]{pr_1_r.png}
			\end{center}			
			Эта вкладка содержит текстовую метку где указано на какой эталон больше всего похож данный образец.При этом в текстовом поле подробно указаны результаты работы алгоритма на каждой итерации сравнения с каждым эталоном :
			\begin{list}{\labelitemi}{\leftmargin=3cm}
					\item значения переменных ${a,b,g,h}$
					\item значения всех функций сходства на данном наборе переменных ${a,b,g,h}$
					\item текущий максимум функций сходства
					\item глобальный максимум функций сходства для всех эталонов
			\end{list}
			В начале работы приложения эта вкладка пуста :
			\begin{center}
				\includegraphics[scale=0.594]{p1_blank.png}
			\end{center}			
	\end{flushleft}
 	\begin{flushleft}
 		\textit{\textbf{Протокол работы :}}
			\\
			\medskip
			\hangindent=1.5cm \hangafter=0 \noindent
			Для проверки корректности работы приложения проведем 4 тестовых прогона :
			\begin{enumerate}
				\item \textit{Граничные случаи :}
				\\
				\medskip
				Как было выяснено при анализе функции сравнения Кульжинского, эта функция возвращает неопределенность вида ${\frac{n}{0}}$ в случае если характеристический ветор образца содержит только 1 или 0.Проверим это:
			\begin{center}
				\includegraphics[scale=0.75]{blank.png}
			\end{center}
			В результате как уже было сказано функция Кульжинского вернет деление на ноль для всех эталонов :
			\begin{center}
				\includegraphics[scale=0.5]{a_r.png}
			\end{center}
				\item \textit{1 прогон :}
				\\
				\medskip Имеем следующий образец :
				\begin{center}
				\begin{tabular}{|c|c|c|c|c|}
					\hline \textbf{Частное} & \textbf{Глобальное} & \textbf{Высокоур.} & \textbf{Админ.} & \textbf{Контроль 	доступа} \\ 
					\hline Да  & Нет & Да  & Да & Нет \\ 
					\hline 
				\end{tabular} 
				\end{center}
				Результат :
				\begin{center}
					\includegraphics[scale=0.6]{ps-1-r.png}
				\end{center}
				\item \textit{2 прогон :}
				\\
				\medskip Имеем следующий образец :
				\begin{center}
				\begin{tabular}{|c|c|c|c|c|}
					\hline \textbf{Частное} & \textbf{Глобальное} & \textbf{Высокоур.} & \textbf{Админ.} & \textbf{Контроль 	доступа} \\ 
					\hline Да  & Нет & Нет & Да & Нет \\ 
					\hline 
				\end{tabular} 
				\end{center}
				Результат :
				\begin{center}
					\includegraphics[scale=0.62]{ps-2-r.png}
				\end{center}
				\item \textit{3 прогон :}
				\\
				\medskip Имеем следующий образец :
				\begin{center}
				\begin{tabular}{|c|c|c|c|c|}
					\hline \textbf{Частное} & \textbf{Глобальное} & \textbf{Высокоур.} & \textbf{Админ.} & \textbf{Контроль 	доступа} \\ 
					\hline Да  & Да & Нет  & Да & Нет \\ 
					\hline 
				\end{tabular} 
				\end{center}																			Результат :
				\begin{center}
					\includegraphics[scale=0.622]{ps-3-r.png}
				\end{center}				
			\end{enumerate}
			\newpage
			\begin{flushleft}
		\begin{center}
\textit{Блок - схемы основных алгоритмических конструкций}
\end{center}
				\begin{enumerate}
					\item Процедура поиска максимального элемента в массиве (метод \textit{getMax()} класса \textit{MainWindow}) :
					\begin{center}
						\includegraphics[scale=1]{scr1.png}
					\end{center}
					\medskip
					\begin{center}
						\includegraphics[scale=0.6]{dia1.png}
					\end{center}
					\newpage
					\item Процедура, повторяемая на каждой итерации алгоритма (метод \textit{DisplayDiff()} \textit{класса MainWindow}):
					\begin{center}
						\includegraphics[scale=0.8]{scr2.png}
					\end{center}
					\begin{center}
						\includegraphics[scale=0.6]{dia2.png}
					\end{center}
					\newpage
					\item Процедура, реализующая основную логику приложения (обработчик кнопки на форме) :
					\begin{center}
						\includegraphics[scale=0.375]{diam.png}
					\end{center}					
															 
				\end{enumerate}
			\end{flushleft}				
	\end{flushleft}		
\end{document}