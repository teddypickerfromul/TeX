\documentclass[11pt,a4paper,sans]{moderncv}
\usepackage[russian]{babel}

\moderncvstyle{casual}                        
\moderncvcolor{green}                         
\usepackage[utf8]{inputenc}                   
\usepackage[scale=0.75]{geometry}


\firstname{Юрий}
\familyname{Кузнецов}
\mobile{+7~(904)~183~9260}                     
\email{teddypickerfromul@gmail.com}          
\homepage{https://github.com/teddypickerfromul}  
\begin{document}
\makecvtitle

\section{Образование}
\cventry{2009-2015}{очная форма обучения}{УлГУ}{Ульяновск}{\textit{инженер САПР}}{Факультет Математики и Информационных технологий, \\
Моделирование и исследование операций в организационно-технических системах  
специалитет, 3 курс}
\cventry{2011-2015}{очная форма обучения}{УлГУ}{Ульяновск}{слушатель}{Российско-Германский факультет \\1 курс}

\section{Резюме}
\cvitem{Вакансия}{\emph{Стажер PHP-разработчик}}
\cvitem{Цели стажировки}{получение опыта участия в реальных проектах и углубление знаний в области back-end веб-разработки и улучшение стиля написание кода}

\section{Иностранные языки}
\cvitemwithcomment{Английский}{базовый уровень}{чтение тех. документации}
\cvitemwithcomment{Немецкий}{начальный уровень}{}

\section{Комьютерные навыки и знания}
\cvitem{Общие качества}{умение работать в команде, понимание основных принципов разработки и поддержки программных продуктов}
\cvitem{Front-end}{HTML, CSS (без опыта сложной верстки), JavaScript(JQuery), базовые представления о user experience в веб-проектах}
\cvitem{Back-end}{PHP на уровне учебных задач, основы SQl}
\cvitem{Прочие языки}{Python на уровне парсинга сайтов и написании небольших скриптов для администраторских целей, Java на уровне учебых задач, представление о архитектуре Android-приложений}
\cvitem{ОС и прикладной инструментарий}{Понимание и использование *nix ОС, основная - Ubuntu;bash-скриптинг и основные консольные инструменты - grep,tail и прочие; 
Eclipse, основы make и doxygen,GIMP, imagemagick, \LaTeX}

\renewcommand{\listitemsymbol}{-~}            

\end{document}
