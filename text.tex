\documentclass[a4paper,12pt]{article}
\usepackage[warn]{mathtext} 
\usepackage[russian]{babel} 
\usepackage[utf8x]{inputenc}
\usepackage{graphicx}
\usepackage[top=1cm,bottom=1.5cm,left=1cm,right=1cm]{geometry}
\usepackage{multirow}

\begin{document}
\begin{titlepage}
\begin{center} 
\large Ульяновский Государственный Университет\\[4.5cm] 

\huge Доклад по экономике\\[0.6cm] 
\large на~тему <<Методы экономической оценки инвестиционных проектов>>\\[3.7cm]

\begin{minipage}{0.5\textwidth} 
\begin{flushleft} 
\emph{Автор:} Кузнецов Юрий Игоревич\\
\emph{Группа:} МОТС 31\\
\emph{Факультет:} ФМиИТ\\
\end{flushleft}
\end{minipage} 
\vfill 
{\large \today} 
{\large  \\ \LaTeX} 
\end{center} 
\thispagestyle{empty}
\end{titlepage} 

\newpage
\tableofcontents

\newpage
	\section{Основные понятия}
		\subsection{Введение}
		Переход России к рыночной системе экономических отношений по-
рождает множество, связанных с этим проблем, среди которых одно из
главенствующих положений занимают проблемы инвестирования. Без
создания заинтересованности потенциальных инвесторов в расширении
объемов вложений в отечественную экономику в принципе невозможно
решить задачи формирования устойчивых хозяйственных связей, развития
производства, повышения благосостояния граждан, возрождения авторитета страны на мировой арене. В связи с этим, особое значение приобретают различные методы оценки и обоснования инвестиционных проектов
и решений, применяемых индивидуальными, корпоративными и институциональными инвесторами. Корректное использование этих методов,
осознанный выбор направлений инвестирования обеспечивают достаточно достоверную оценку ожидаемых последствий их реализации.\\
Для развитых стран мира и стран, развивающихся или переживающих кризис в экономике, инвестиционная политика, стимулирующая принятие инвестиционных решений и исполнение инвестиционных проектов,
обеспечивающих поддержание технико-технологического состояния на-
родного хозяйства на мировом уровне или достижения его, является од-
ним из важнейших средств успешного осуществления независимой внутренней и внешней политики. В этом состоит общеэкономическое значение
инвестиций и принятия инвестиционных решений.\\
Для отдельных инвесторов значение инвестиционных решений, необходимость тщательного и корректного их обоснования определяются
рядом условий.\\
Во-первых, принятие того или иного инвестиционного решения, выбор соответствующего проекта предполагают, что в процессе его реализации инвестор связывает свои материальные и финансовые ресурсы на достаточно длительный период. Начало всякого инвестиционного проекта
связано с материальными и финансовыми расходами. Получаемые при
этом средства не могут быть мгновенно обращены в денежный капитал.
Во-вторых, приступая к некоторому долгосрочному инвестиционному проекту, инвестор формирует технико-технологический уровень своего
производства и выпускаемой продукции на достаточно продолжительный
период и, таким образом, определяет свои перспективы в будущей конкурентной борьбе с производителями аналогичной продукции. В процессе
инвестирования, у инвестора может не оказаться достаточно свободных
средств на запуск в производство новых изделий и разработок. Это повышает требования к обоснованию технико-технологических решений, закладываемых в инвестиционный проект и обеспечивающих конкурентоспособность данного инвестора.\\
В-третьих, некоторые инвесторы еще в процессе своей деятельности
должны обеспечить выполнение принятых долгосрочных обязательств перед клиентами, вкладчиками и т. д., что повышает требования к качеству
обоснования инвестиционных решений.\\
В-четвертых, принимаемые инвестиционные решения носят обычно
комплексный характер и охватывают практически все стороны деятельности предприятия как действующего, так и создаваемого. Внедрение инвестиционного проекта оказывает влияние на процессы стратегического
планирования фирмы, управление маркетингом и сбытом, связи с поставщиками и обеспечение необходимым сырьем и материалами, финансирование, оперативное управление производством и т. п.\\
Таким образом, процесс принятия инвестиционных решений должен
отражать как их последствия, так и возможности инвестора.
Так что же следует понимать под инвестицией и соответственно под
инвестиционным решением? В наиболее широком смысле слово «инвестировать» означает: «расстаться с деньгами сегодня, чтобы получить
большую их сумму в будущем». Два фактора обычно связаны с данным
процессом – время и риск. Отдавать деньги приходится сейчас и в определенном количестве. Вознаграждение поступает позже, если поступает вообще, и его величина заранее неизвестна.\\
Под инвестиционной деятельностью любого инвестора, связанной с
исполнением конкретной инвестиции, понимается определенная последовательность его поступков или действий, направленных на достижение
поставленной им цели и включающих в себя обоснование и финансирование создания объекта, его производительное, или полезное использование,
реализацию возможностей, которые появляются в процессе эксплуатации
данного объекта, и ликвидацию (продажу) или уничтожение его, если использование этого объекта препятствует достижению целей инвестора.
Иными словами, инвестиционная деятельность – долгосрочные вложения денежных средств и совокупность действий по реализации инвестиционных проектов.

		\subsection{Понятие инвестиции с различных точек зрения}
		Многообразие понятий инвестиции в значительной степени определяется широтой сущностных сторон этой сложной экономической категории.Поэтому необходимо рассмотреть основные характеристики, формирующие ее сущность : 
\begin{flushleft}
 			\hangindent=1.5cm \hangafter=0 \noindent
			\begin{enumerate}
				\item \textbf{\textit{Инвестиции как объект экономического управления}}\\ \medskip
				Предметная сущность инвестиций непосредственно связана с экономической сферой её проявления.Инвестиции трактуются всеми исследователями как категория экономическая, хотя и связанная с технологическими, социальными, природоохранными и иными аспектами их осуществления,Иными словами, категория "инвестиции" входит в понятийно-категориальный аппарат, связанный со сферой экономических отношений, экономической деятельности.То есть инвестиции являются субъектом экономического управления как на макро-, так м на микроуровне любых   
экономических систем.
				\item \textbf{\textit{Инвестиции как наиболее активная  форма вовлечения накопленного капитала в экономический процесс}}\\ \medskip
			В теории инвестиций их связь с накопленным капиталом (сбережениями) занимает центральное место.Это определяется сущностной природой капитала как экономического ресурса, предназначенного к инвестированию.Только путём инвестирования капитал как накопленная ценность вовлекается в экономический процесс.Часть денежного или иного капитала в силу требований ликвидности представляет собой форму страхового резерва, обеспечивающего ритмичность хозяйственной деятельности, платёжеспособность и тому подобное, сохраняя пассивную формую.\\
			Уровень потребления накопленного капитала как инвестиционного ресурса, вовлекаемого в реальный производственный процесс предприятия, имеет минимальные экономические границы, определяемые предельным продуктом капитала и нормой	амортизации капитала в производственный процессе.
				\item \textbf{\textit{Инвестиции как возможность использования накопленного капитала во всех альтернативных его формах}}\\ \medskip
			В инвестиционном процессе каждая из форм накопленного капитала имеет свой диапазон возможностей и специфику механизмов конкретного использования.\\
			Наиболее универсальной с позиции сферы использования является денежная форма капитала, которая часто трансформируется в иные формы.Капитал, накопленный в форме запасов конкретных материальных и нематериальных благ имеет более узкое функциональное значение.\\
			Используемый в инвестиционном процессе капитал во всех его формах может быть задействован прежде всего в производственной деятельности предприятия.С этой позиции капитал выступает как фактор производства.\\
			При этом в процессе производства продукции инвестируемый капитал не является самодостаточным  фактором, а используется в комплексе с другими экономическими ресурсами(например с трудовыми ресурсами).
			\newpage       
				\item \textbf{\textit{Инвестиции как альтернативная возможность вложения капитала в любые объекты хозяйственной деятельности }}\\ \medskip
			Инвестируемый предприятием капитал целенаправленно вкладывается в формирование имущества предприятия, предназначенного для осуществления различных форм его хозяйственной деятельности и производства различной продукции.При этом из обширного диапазона возможных объектов инвестирования капитала предприятие самостоятельно определяет приоритетные формы имущественных ценностей, называемых активами.\\
			С позиции возможностей вложения капитала инвестиции характеризуются как комбинаторные процессы.      				 							\item \textbf{\textit{Инвестиции как источник генерирования предпринимательской деятельности}}\\ \medskip
			Целью инвестирования является достижение конкретного заранее предопределённого эффекта, который может носить как экономический, так и неэкономический характер.На уровне предприятий приоритетной целевой установкой инвестиций является достижение, как правило, экономического эффекта, который может быть получен в форме прироста суммы инвестированного капитала, положительной величины  инвестиционной прибыли, положительной величины чистого денежного потока, обеспечения сохранения ранее вложенного капитала и т.п.\\
			Достижение экономического эффекта инвестиций определяется их потенциальной способностью генерировать доход. 	 
				\item \textbf{\textit{Инвестиции как объект рыночных отношений}}\\ \medskip			 
			Различные виды инвестиций формируют особый вид рынка - инвестиционный рынок, характеризуемый спросом, предложением, ценой а также совокупностью определённых субъектов рыночных отношений, тесно связанный с остальными рынками.\\
			\textit{Предложение} инвестиционных ресурсов, товаров и инструментов исходит от предприятий-производителей капитальных товаров, собственников недвижимости, владельцев нематериальных активов, эмитентов, разнообразных финансовых институтов.\\
			\textit{Спрос} на инвестиционные ресурсы, товары и инструменты предприятия предъявляют для реализации своей инвестиционной стратегии в сфере реального и финансового инвестирования.\\
			\textit{Цена} на инвестиционные товары и инструменты в системе рыночных отношений формируется с учётом их инвестиционной привлекательности под воздействием спроса и предложения.Ценой инвестиционных ресурсов выступает обычно ставка процента, формируемая на рынке капитала.  
				\item \textbf{\textit{Инвестиции как объект собственности и распоряжения}}\\ \medskip
			Как объект предпринимательской деятельности инвестиции являются носителем прав собственности и распоряжения.\\
			Инвестиционный капитал как объект собственности может выступать носителем всех форм этой собственности(индивидуальной, частной, муниципальной и т.п.).Является пассивным объектом.\\
			Инвестиционный капитал как объект распоряжения может выступать во всех разрешённых законодательством формах этого распоряжения.Носителем прав распоряжения может выступать при этом как финансовый, так и реальный капитал.Является активным объектом.\\   				   
			Соотношение собственного и заёмного капиталов называется \textit{структурой} капитала.\\ \medskip
			\end{enumerate}
\end{flushleft}			 
			Также инвестиции можно рассматривать как факторы риска, находящийся в прямой зависимости от уровня ожидаемой доходности, и как носитель фактора ликвидности.
			\newpage
			В соответствии со всем вышеизложенным можно дать наиболее общее определение понятию инвестиции :\\ 
			
\textbf{\textit{Инвестиции предприятия представляют собой вложение капитала во всех его формах в различные объекты (инструменты) его хозяйственной деятельности с целью получения прибыли, а также достижения иного экономического или внеэкономического эффекта, осуществление которого базируется на рыночных принципах и связано с факторами времени, риска и ликвидности}}  								
	\subsection{Виды эффективности инвестиционных проектов}
Международная практика оценки эффективности инвестиций базируется на концепции временной стоимости денег и основана на ряде
принципов:\\ \medskip
			\begin{enumerate}
				\item Эффективность использования инвестируемого капитала оценивается путём сопоставления денежного потока (\textit{cash flow}), который формируется в процессе реализации инвестиционного проекта и исходной инвестиции. Проект признается эффективным, если обеспечивается возврат
исходной суммы инвестиций и требуемая доходность для инвесторов,
предоставивших капитал;
\item Инвестируемый капитал, равно как и денежный поток, приводится
к настоящему времени или к определённому расчётному году (который,
как правило, предшествует началу реализации проекта);
\item Процесс дисконтирования капитальных вложений и денежных по-
токов осуществляется по различным ставкам дисконта, которые определяются в зависимости от особенностей инвестиционных проектов. При
определении ставки дисконта учитываются структура инвестиций и стоимость отдельных составляющих капитала.\\
\end{enumerate}

Эффективность проекта характеризуется системой показателей, отражающих соотношение затрат и результатов применительно к интересам
его участников.
Различают следующие показатели эффективности инвестиционного
проекта :
\begin{itemize}
\item показатели коммерческой (финансовой) эффективности, учитывающие финансовые последствия реализации проекта для его непосредственных участников
\item показатели бюджетной эффективности, отражающие финансовые последствия осуществления проекта для федерального, регионального или местного бюджета
\item показатели экономической эффективности, учитывающие затраты и
результаты, связанные с реализацией проекта, выходящие за пределы
прямых финансовых интересов участников инвестиционного проекта и
допускающие стоимостное измерение.\\
\end{itemize}
В процессе разработки проекта производится оценка его социальных
и экологических последствий, также затрат, связанных с социальными мероприятиями и охраной окружающей среды.
Оценка предстоящих затрат и результатов при определении эффективности инвестиционного проекта осуществляется в пределах расчётного периода, продолжительность которого (горизонт расчёта) принимается с учётом :
\begin{itemize}
	\item продолжительности создания, эксплуатации и (при необходимости) ликвидации объекта;
	\item средневзвешенного нормативного срока службы основного технологического оборудования;
	\item достижения заданных характеристик прибыли (массы и/или нормы прибыли и т. д.);
\item требований инвестора. \\
\end{itemize}
Горизонт расчёта измеряется количеством шагов расчёта. Шагом расчёта при определении показателей эффективности в пределах расчётного периода могут быть: месяц, квартал или год.
Затраты, осуществляемые участниками подразделяются на первоначальные (капиталообразующие инвестиции), текущие и ликвидационные,
которые осуществляются соответственно на стадиях строительной, функционирования и ликвидационной. Для стоимостной оценки результатов затрат могут использоваться базисные, мировые, прогнозные и расчётные цены.\\
При оценке эффективности инвестиционного проекта соизмерение
разновременных показателей осуществляется путём приведения (дисконтирования) их к ценности в начальном периоде. Для приведения разновременных затрат, результатов и эффектов используется норма дисконта ${E}$, равная приемлемой для инвестора норме дохода на капитал.\\ 

Техническое приведение к базисному моменту времени затрат, результатов и эффектов, имеющих место на ${t}$-ом шаге расчёта реализации проекта, удобно производить путём их умножения на коэффициент дисконтирования $\alpha_t$, определяемый для постоянной нормы дисконта ${E}$ по формуле:
\begin{equation}
		\alpha_t = \frac{1}{(1+E)^t}
\end{equation}
К основным принципам оценки эффективности инвестиций, приме-
имым ко всем типам инвестиционных проектов также относятся:
\begin{enumerate}
\item рассмотрение проекта на протяжение всего его жизненного цикла
(расчетного периода);
\item моделирование денежных потоков за расчетный период с учетом
возможности использования различных валют;
\item сопоставимость условий сравнения различных проектов (вариан-
тов);
\item обеспечение положительности и максимума эффекта;
\item учёт фактора времени:
	\begin{itemize}
\item разбиение расчётного периода на шаги. При этом
необходимо учитывать: цель расчёта; продолжительность
различных фаз жизненного цикла, чтобы основные этапы
совпадали с началом шага; изменение цен в течение шага. Рекомендуется изменение в пределах 5 – 10 \% ;обозримость выходной информации, удобство её оценки;\\
\item учёт динамики. Необходимо учитывать изменение
во времени параметров проекта, а также разрывы во времени
между производством продукции и поступлением ресурсов,
производством продукции и ее продажей;
	\end{itemize}
\item
 сравнение полученных данных по проекту с ситуацией без
проекта;
\item
 учет всех наиболее существенных последствий проекта;
\item
 учет наличия разных участников проекта, несовпадения их ин-
тересов, различных оценок стоимости капитала;
\item
 многоэтапность оценки;
\item
 учет влияния потребности в оборотном капитале;
\item
 учет влияния инфляции и возможности использования не-
скольких валют;
\item
 учет влияния неопределенности и риска.
\end{enumerate}
\newpage
Осуществление эффективных проектов увеличивает поступающий в
распоряжение общества внутренний валовой продукт (ВВП), который затем делится между участвующими в проекте субъектами (фирмами, акционерами и работниками, банками, бюджетами разных уровней и пр.).
Поступлениями и затратами этих субъектов определяются различные виды эффективности инвестиционного проекта.
Рекомендуется оценивать следующие виды эффективности :
\begin{itemize}
\item эффективность проекта в целом
\item эффективность участия в проекте
\end{itemize}
\textit{Эффективность проекта в целом} оценивается с целью определения потенциальной привлекательности проекта для возможных участников и поисков источников финансирования. Она включает в себя \textit{общественную} (социально-экономическую) эффективность проекта и коммерческую эффективность проекта.
Показатели общественной эффективности учитывают социально-экономические последствия осуществления инвестиционного проекта для общества в целом, в т. ч. как непосредственные результаты и затраты
проекта, так и «внешние»: затраты и результаты в смежных секторах
экономики, экологические, социальные и иные внеэкономические эффекты.\\
Показатели коммерческой эффективности проекта учитывают финансовые последствия его осуществления для единственного участника, реализующего инвестиционный проект, в предположении, что он производит все необходимые для реализации проекта затраты и пользуется всеми его результатами.\\
Показатели эффективности проекта в целом характеризуют с
экономической точки зрения технические и организационные проектные решения.\\
\textit{Эффективность проекта в целом} определяется с целью проверки
реализуемости инвестиционного проекта и заинтересованности в нем всех
его участников.\\
Эффективность участия в проекте включает:
\begin{itemize}
\item эффективность для предприятий-участников;
\item эффективность инвестирования в акции предприятия (эффективность для акционеров);
\item эффективность участия в проекте структур более высокого уровня
по отношению к предприятиям-участникам инвестиционного проекта, в
т. ч. региональную и народнохозяйственную эффективность (для отдельных регионов и народного хозяйства РФ), отраслевую эффективность
(для отдельных отраслей народного хозяйства, финансово-промышленных групп, объединений предприятий и холдинговых структур);
\item бюджетную эффективность инвестиционного проекта (эффективность участия государства в проекте с точки зрения расходов и доходов бюджетов всех уровней).
\end{itemize}
Коммерческая эффективность (финансовое обоснование) проекта
определяется соотношением финансовых затрат и результатов, обеспечивающих требуемую норму доходности. В качестве эффекта на ${t}$-м шаге
выступает \textit{сальдо} реальных денег.\\
При осуществлении проекта выделяют три вида деятельности: инвестиционную, операционную и финансовую. В рамках каждого вида деятельности происходит приток и отток денежных средств.\\
Потоком реальных денег называется разность между притоком и от-
током денежных средств от инвестиционной и операционной деятельности в каждом периоде осуществления проекта.\\
\textit{Сальдо} реальных денег называется разность между притоком и оттоком денежных средств от всех трёх видов деятельности.
Необходимым критерием принятия инвестиционного проекта является положительное сальдо накопленных реальных денег в любом временном интервале, где данный участник осуществляет затраты или получает доходы. Отрицательная величина сальдо накопленных реальных денег свидетельствует о необходимости привлечения участником дополнительных собственных или заёмных средств и отражения этих средств в расчётах эффективности.
\newpage
Для стоимостного выражения денежных потоков могут применяться
следующие виды цен:
\begin{flushleft}
				\begin{enumerate}
					\item текущие цены;
					\item прогнозные цены (с учётом инфляции);
					\item дефлированные цены. Это прогнозные цены, приведённые 						к уровню цен фиксированного момента времени путём 								деления на общий (базисный) индекс инфляции.
				\end{enumerate}
\end{flushleft}
При расчёте показателей экономической эффективности на уровне
народного хозяйства в состав результатов проекта включают (в стоимостном выражении) :
\begin{itemize}
\item конечные производственные результаты (выручка от продажи,
выручка от реализации имущества и интеллектуальной собственности);
\item социальные и экологические результаты, рассчитанные исходя
из совместного воздействия всех участников проекта на здоровье населения, социальную и экологическую обстановку в регионах;
\item прямые финансовые результаты;
\item кредиты и займы иностранных государств, банков и фирм, поступления от импортных пошлин и т. п.
\end{itemize} 
\subsection{Вспомогательные понятия}
Для анализа инвестиций используются следующие понятия :\\

\textbf{\textit{Сaльдо}} - разность между поступлениями и расходами за определённый промежуток времени.\\

\textit{\textbf{Операционная деятельность}} - деятельность организации, преследующая извлечение прибыли в качестве основной цели, либо не имеющая извлечение прибыли в качестве такой цели в соответствии с предметом и целями деятельности, т. е. производством промышленной продукции, выполнением строительных работ, сельским хозяйством, продажей товаров, оказанием услуг общественного питания, заготовкой сельскохозяйственной продукции, сдачей имущества в аренду и др.\\

\textbf{\textit{Дисконтирование}} - это приведение всех денежных потоков в будущем (потоков платежей) к единому моменту времени в настоящем. Дисконтирование является базой для расчётов стоимости денег с учётом фактора времени.\\
Приведение к моменту времени в прошлом называют дисконтированием.
Приведение к моменту в будущем называют наращением (\textit{компаундированием}).
Дисконтирование выполняется путём умножения будущих денежных потоков (потоков платежей) на коэффициент дисконтирования : ${k_d}$:
\begin{equation}
	k_d = \frac{1}{(1+i)^n}
\end{equation}
где ${i}$ -  процентная ставка, ${n}$ - номер периода.
\newpage	    
	\section{Методы оценки экономической эффективности инвестиций}
	Методы оценки экономической эффективности инвестиций делятся
на три группы:
\begin{itemize}
\item статические;
\item динамические;
\item методы оценки эффективности в условиях неопределённости и
риска.
\end{itemize}

		\subsection{Статические методы}
		Статические методы в свою очередь делятся на \textit{одно- и многопериодные} статические методы.\\
		\textit{Однопериодные} статические методы основаны на сравнении вариантов инвестиционных проектов не за весь проектный период, а за 1
год, в качестве которого выбирается первый год работы предприятия на
полную проектную мощность. Предпочтительный вариант выбирается по
установленным критериям:
\begin{itemize}
\item объем инвестиционных затрат;
\item прибыль;
\item доход от проекта;
\item рентабельность
\end{itemize}
\textit{Многопериодные} статические методы используются для оценки
эффективности инвестиционных проектов, имеющих различные периоды
осуществления. В их основе лежит метод оценки и оптимизации срока
окупаемости (${CO}$, ${PP}$), который при равномерном поступлении прибыли (${P_t}=const$) ) рассчитывается следующим образом:
\begin{equation}
	PP = \frac{IC}{P_t}
\end{equation}
где ${IC}$ - объем инвестиций, ${P_t}$ - чистая прибыль, ${PP}$ - период окупаемости.\\
Если прибыль неравномерна по годам, то находится момент, когда
сумма эффектов (прибыли) равна сумме инвестиций:
\begin{equation}
	\sum P_t = \sum CI_t
\end{equation}
Достоинством этого метода является простота расчётов. В качестве
недостатка можно отметить невозможность учесть динамику результатов
после того, как проект окупит себя.\\
\begin{center}
	\textbf{Показатель простой рентабельности инвестиций}
\end{center}

Показатель расчетной нормы прибыли \textit{\textbf{ARR (Accounting Rate of
Return)}} является обратным по содержанию сроку окупаемости капитальных вложений.
Расчётная норма прибыли отражает эффективность инвестиций в
виде процентного отношения денежных поступлений к сумме первоначальных инвестиций:
\begin{equation}
	ARR = \frac{{P_t}\prime}{IC}	
\end{equation}
где ${ARR}$ - расчётная норма прибыли инвестиции, ${{P_t} \prime}$ - среднегодовая прибыль, ${IC}$ - объем инвестиций.\\
Этому показателю присущи все недостатки, свойственные показате-
лю срока окупаемости. Он принимает в расчёт только два критических ас-
пекта, инвестиции и денежные поступления от текущей хозяйственной
деятельности и игнорирует продолжительность экономического срока
жизни инвестиций.\\
Использование ARR в настоящее время во многих фирмах и странах
мира объясняется рядом достоинств этого показателя. Во-первых, он
прост и очевиден при расчёте, а также не требует использования таких
изощрённых приемов, как дисконтирование денежных потоков.\\
Во-вторых, показатель ARR удобен для встраивания его в систему
стимулирования руководящего персонала фирм. Именно поэтому те фирмы, которые увязывают системы поощрения управляющих своих филиалов и подразделений с результативностью их инвестиций, обращаются ARR. Это позволяет задать руководителям среднего звена легко понимаемую ими систему ориентиров инвестиционной деятельности.\\
Недостатки показателя расчётной рентабельности инвестиций являются оборотной стороной его достоинств.\\
Во-первых, так же, как показатель периода окупаемости, ARR не учитывает разной ценности денежных
средств во времени, поскольку средства, поступающие, например, на 10-й
год после вложения средств, оцениваются по тому же уровню рентабельности, что и поступления в первом году.\\ 
Во-вторых, этот метод игнорирует различия в продолжительности эксплуатации активов, созданных благодаря инвестированию. В-третьих, расчёты на основе ARR носят более «витринный» характер, чем расчёты на основе показателей, использующих данные о денежных потоках. Последние показывают реальное изменение ценности фирмы в результате инвестиций, тогда как ARR ориентирована преимущественно на получение оценки проектов, адекватной ожиданиям и требованиям акционеров и других лиц и фирм «со стороны».
\begin{center}
	\textbf{\textit{Максимальный денежный отток (Cash Outflow)}}
\end{center}
\textbf{\textit{Максимальный денежный отток (Cash Outflow)}}, называемый в отечественных источниках \textit{потребностью финансирования (ПФ)} – это максимальное значение абсолютной величины отрицательного накопленного
сальдо от инвестиционной и операционной деятельности. Величина потребности финансирования показывает минимальный объем внешнего
финансирования проекта, необходимый для обеспечения его финансовой
реализуемости.Поэтому потребность финансирования называют ещё капиталом риска.\\
Термин внешнее финансирование в отличие от внутреннего предполагает любые источники финансирования (собственные и привлечённые),
внешние по отношению к проекту, тогда как внутреннее финансирование
осуществляется в процессе реализации проекта за счёт получения чистой
прибыли и амортизационных отчислений.
		\medskip
		\begin{center}
			\includegraphics[scale=0.38]{econom.png} 
		\end{center}
\subsection{Динамические методы}
Динамические методы основаны на изменении стоимости денег во
времени и учёте влияния временного фактора. При расчёте эффективности
фактор времени нужно учитывать из-за:
\begin{enumerate}
\item динамичности технико-экономических показателей предприятия, проявляющейся в изменении объёмов и структуры продукции,
норм расхода сырья, материалов, численности персонала, длительности
производственного цикла. Данные изменения особенно сильно проявляются в период освоения мощностей или технических перевооружений.
Учёт данных изменений производится путём формирования денежных
потоков с учётом особенностей процесса производства на каждом шаге
расчётного периода;
\item физического износа основных фондов, что приводит к снижению их производительности и увеличению затрат на содержание, эксплуатацию и ремонт. Физический износ учитывается при формировании
производственной программы, операционных издержек, сроков замены
оборудования;
\item изменение во времени цен на производимую продукцию и потребляемые ресурсы;
\item несовпадения объёмов выполняемых строительно-монтажных работ с размерами оплаты этих работ;
\item разновременности затрат результатов и эффектов;
\item изменения во времени экономических нормативов;
\item разрывов во времени, лагов между производством и реализацией продукции, между оплатой и потреблением ресурсов.
\end{enumerate}
Сравнение различных инвестиционных проектов (или вариантов
проекта) и выбор лучшего из них рекомендуется производить с использование различных показателей, к которым относятся:
\begin{itemize}

\item чистый дисконтированный доход (ЧДД) или интегральный эффект;
\item индекс доходности (ИД);
\item внутренняя норма доходности (ВНД);
\item срок окупаемости (СО);
\item другие показатели, отражающие интересы участников или специфику проекта.
\end{itemize}
\begin{center}
	\textbf{\textit{Чистый дисконтированный доход}}
\end{center}
\textbf{\textit{Чистый дисконтированный доход (ЧДД})} определяется как сумма
текущих эффектов за весь расчётный период, приведённая к начальному
шагу, или как превышение интегральных результатов над интегральными
затратами.
Если в течение расчётного периода не происходит инфляционного
изменения цен или расчёт производится в базовых ценах, то величина
ЧДД для постоянной нормы дисконта вычисляется по формуле:
\begin{equation}
	Э_{ИНТ} = ЧДД = \sum_{t=0}^ T (R_t - З_t)\cdot \frac{1}{(1+E)^t}
\end{equation}
где ${R_t}$ – результаты, достигаемые на ${t}$-ом шаге расчёта; ${З_t}$ – затраты, осуществляемые на том же шаге; ${Т}$ – горизонт расчёта; ${Э_t = (R_t – З_t)}$ – эффект, достигаемый на ${t}$ -м шаге.\\
Если ЧДД инвестиционного проекта положителен, проект является
эффективным (при данной норме дисконта) и может рассматриваться вопрос о его принятии. Чем больше ЧДД, тем эффективнее проект.
Если инвестиционный проект будет осуществлён при отрицательном
ЧДД, инвестор понесет убытки, т. е. проект неэффективен.
\newpage
На практике часто пользуются модифицированной формулой для
определения ЧДД. Для этого из состава ${З_t}$ исключают капитальные вложения и обозначают через :\\
${К_t}$ – капиталовложения на ${t}$-ом шаге;\\
${К}$ – сумму дисконтированных капиталовложений, т. е.
\begin{equation}
	K = \sum_{t=0}^ T K_t \cdot \frac{1}{(1+E)^t}
\end{equation}
а через ${З_t \prime}$ – затраты на ${t}$-ом шаге при условии, что в них не входят капиталовложения.Тогда формула для ЧДД записывается в виде:
\begin{equation}
	ЧДД = {\sum_{t=0}^ T \frac{(R_t - {З_t} \prime)}{(1+E)^t}} - K
\end{equation}
и выражает разницу между суммой приведённых эффектов и приведённой
к тому же моменту времени величиной капитальных вложений ${K}$.\\
Наиболее эффективным является применение показателя чистого
дисконтированного дохода в качестве критериального механизма, показывающего минимальную нормативную рентабельность (норму дисконта)
инвестиций за экономический срок их жизни. Если ЧДД является положительной величиной, то это означает возможность получения дополнительного дохода сверх нормативной прибыли, при отрицательной величине
ЧДД прогнозируемые денежные поступления не обеспечивают получения
минимальной нормативной прибыли и возмещения инвестиций. При ЧДД,
близкому к 0, нормативная прибыль едва обеспечивается (но только в
случае, если оценки денежных поступлений и прогнозируемого экономического срока жизни инвестиций окажутся точными).\\
Несмотря на все эти преимущества оценки инвестиций, метод чистого дисконтированного дохода не дает ответа на все вопросы, связанные с
экономической эффективностью капиталовложений. Этот метод дает ответ лишь на вопрос, способствует ли анализируемый вариант инвестирования росту ценности фирмы или богатства инвестора вообще, но никак
не говорит об относительной мере такого роста.
\begin{center}
	\textbf{\textit{Индекс доходности}}
\end{center}
\textbf{\textit{Индекс доходности}} представляет собой отношение суммы приведённых эффектов к величине капиталовложений:
\begin{equation}
	ИД = \frac{1}{K} \cdot {\sum_{t=0}^ T \frac{(R_t - {З_t} \prime)}{(1+E)^t}}
\end{equation}
Индекс доходности тесно связан с ЧДД. Он строится из тех же элементов и его значение связано со значением ЧДД:
\begin{itemize}
\item \textit{Если ЧДД положителен, то ИД >1 и наоборот}.
\item \textit{Если ИД > 1, проект эффективен, если ИД <1 – неэффективен}.
\end{itemize}
\textbf{\textit{Срок окупаемости}} – минимальный временной интервал (от начала осуществления проекта), за пределами которого интегральный эффект
становится и в дальнейшем остается неотрицательным. Иными словами,
это период (измеряемый в месяцах, кварталах или годах), начиная с которого первоначальные вложения и другие затраты, связанные с инвестиционным проектом, покрываются суммарными результатами его осуществления. Срок окупаемости можно определить как отношение инвестиций к
среднегодовому денежному потоку. Результаты и затраты, связанные с
осуществлением проекта можно вычислить с дисконтированием или без
него. Соответственно, получится два различных срока окупаемости. Срок
окупаемости рекомендуется определять с использованием дисконтирования.
\newpage
Показатель срока окупаемости инвестиционного проекта с неравны-
ми из года в год денежными потоками можно разложить на целую ${j}$ и
дробную ${d}$ его составляющие ${СО = j + d}$. Целое значение срока окупаемости находится последовательным сложением чистых денежных по-
токов (дисконтированных денежных потоков) за соответствующие периоды времени до тех пор, пока полученная сумма последний раз будет
меньше величины начальных инвестиционных затрат. То есть целая часть
срока окупаемости – это период, в котором накопленная стоимость де-
нежных потоков принимает своё последнее отрицательное значение, при
этом следует соблюдать следующие неравенства:
\begin{equation}
− K_0  + ДП_1 + ДП_2  + ... + ДП_j  \le 0 , 1 \le j \le n 
\end{equation}
где $K_0$ – начальные инвестиции, ДП – денежные потоки,\\
Дробная часть срока окупаемости определяется по формуле:
\begin{equation}
	d = \frac{|− K_0 + ДП_1 + ДП_2 + \cdots + ДП_j|}{ДП_{j+1}}
\end{equation}
\begin{center}
	\textbf{\textit{Внутренняя норма доходности (ВНД)}}
\end{center}
\textbf{\textit{Внутренняя норма доходности (ВНД)}} представляет собой ту норму дисконта (${E_{вн}}$), при которой величина приведённых эффектов равна
приведённым капиталовложениям. Иными словами ${E_{вн}}$ (ВНД) является
решением уравнения:
\begin{equation}
	\sum_{t=0}^ T \frac{(R_t - {З_t} \prime)}{(1+E_{вн})} = \sum_{t=0}^ T \frac{K_t}{(1+E_{вн})^t}
\end{equation}
Если расчет ЧДД инвестиционного проекта дает ответ на вопрос, яв-
ляется он эффективным или нет при некоторой заданной норме дисконта
${E}$, то ВНД проекта определяется в процессе расчёта и затем сравнивается
с требуемой инвестором нормой дохода на вкладываемый капитал. В случае, когда ВНД равна или больше требуемой инвестором нормы дохода на
капитал, инвестиции в данный инвестиционный проект оправданы, и может рассматриваться вопрос о его принятии. В противном случае инвестиции в данный проект нецелесообразны. Если сравнение альтернативных
инвестиционных проектов по ЧДД и ВНД приводят к противоположным
результатам, предпочтение следует отдавать ЧДД.\\
Для расчёта внутренней нормы доходности также используют упрощённую формулу. Для этого необходимо выбрать две ставки дисконтирования ${E_1 < E_2}$, таким образом, чтобы в интервале (${E_1}$; ${E_2}$) функция
${ЧДД = f(Е)}$ меняла своё значение с «\textbf{+}» на «\textbf{-}» или наоборот. Далее используют формулу:
\begin{equation}
	ВНД = E_1 + \frac{ЧДД(E_1)}{ЧДД(E_1) - ЧДД(E_2)}\cdot (E_2 - E_1)
\end{equation}
Точность вычисления является обратной длине интервала $(E_1; E_2)$.
Поэтому наилучшая аппроксимация достигается в случае, когда длина интервала принимается минимальной (1 \%).\\
Для того чтобы легче разобраться в категории ВНД необходимо ввести допущения, что речь будет идти о таких инвестиционных проектах, при реализации которых:
\begin{itemize}
	\item надо сначала осуществить затраты денежных средств (допустить
отток средств) и лишь потом можно рассчитывать на денежные поступления (притоки средств)
	\item денежные поступления носят кумулятивный характер, причём их
знак меняется лишь однажды (т. е. сначала они могут быть отрицательными, но став затем положительными, будут оставаться такими на протяжении всего расчётного периода)
\end{itemize}
\newpage
Для таких инвестиций справедливо утверждение о том, что чем выше норма дисконта (Е), тем меньше величина интегрального эффекта
(ЧДД) :
\begin{center}
	\includegraphics[scale=0.3]{img1.png} 
\end{center}
Как видно из графика, ВНД – это та величина нормы дисконта Е, при
которой кривая изменения ЧДД пересекает горизонтальную ось, т. е. интегральный экономический эффект (ЧДД) оказывается равным нулю.
Принцип сравнения этих показателей такой:
\begin{itemize}
\item если ВНД > ${E}$ – проект приемлем (т. к. ЧДД в этом случае имеет положительное значение)
\item если ВНД < ${E}$ – проект не приемлем (т. к. ЧДД отрицательна)
\item если ВНД = ${E}$ – можно принимать любое решение
\end{itemize}
Кроме того, этот показатель может служить основой для ранжирования проектов по степени выгодности, при прочих равных условиях, т. е.
при тождественности основных исходных параметров сравниваемых
проектов:
\begin{itemize}
\item равной сумме инвестиций
\item одинаковой продолжительности расчётного периода;
\item равной уровню риска.
\end{itemize}
Внутренняя норма доходности может быть использована также для
экономической оценки проектных решений, если известны приемлемые
значения ВНД (зависящие от области применения) у проектов данного типа; для оценки степени устойчивости инвестиционных проектов по разности ВНД – ${E}$; для установления участниками проекта нормы дисконта ${E}$
по данным о внутренней норме доходности альтернативных направлений
вложения ими собственных средств.\\
Ряд инвестиционных проектов имеет денежные потоки, в которых
инвестиционные затраты возникают на заключительных стадиях существования этих проектов. Этим отрицательным элементам денежного потока
предшествуют положительные величины денежных поступлений.\\
Такие инвестиционные проекты могут иметь две внутренние нормы
доходности или не иметь ни одной. Например, к такому типу относится
следующий денежный поток: \textit{1000; –3000; 2500}.\\
С целью измерения доходности проектов с нетрадиционными де-
нежными потоками целесообразно рассчитывать \textit{модифицированную
норму доходности} \textit{(\textbf{MIRR})}.
Один из способов её расчёта заключается в использовании подхода,
в соответствии с которым \textit{\textbf{MIRR}} является ставкой, уравновешивающей современную стоимость инвестиций данного проекта и конечную (терминальную) стоимость поступлений. При этом искомый показатель ставки
доходности является неизвестной величиной в следующем уравнении:
\begin{equation}
	MIRR = \sqrt[T]{\frac{TV(P)}{-PV(I)}}=1
\end{equation}
где :
\begin{equation}
	TV(P) = \sum_{t=0}^ T P_t \cdot {(1+E)^{T-t}}\\
\end{equation}
\begin{equation}
	PV(I) = \sum_{t=0}^ T \frac{I_t}{(1+E)^t}
\end{equation}
где ${I_t}$ – инвестиции в году ${t}$; ${P_t}$ – доходы, получение которых предполагается в году ${t}$; ${E}$ – ставка дисконтирования; ${PV(I)}$ – суммарная современная стоимость инвестиций; ${TV(P)}$ – суммарная конечная (терминальная) стоимость поступлений.\\\medskip

При необходимости учёта инфляции формулы должны быть преобразованы так, чтобы из входящих в них значений затрат и результатов было исключено инфляционное изменение цен, т. е. чтобы величины критериев были приведены к ценам расчётного периода. Это можно выполнить введением прогнозных индексов и дефлирующих множителей.\\
Наряду с перечисленными критериями, в ряде случаев возможно использование и ряда других показателей: интегральной эффективности за-
трат, точки безубыточности, простой нормы прибыли, капиталоотдачи и
т. д. Для применения каждого из них необходимо ясное представление о
том, какой вопрос экономической оценки проекта решается с его использованием и как осуществляется выбор решения.\\
Ни один из перечисленных критериев не является сам по себе достаточным для принятия проекта. Решение об инвестировании средств в
проект должно приниматься с учётом значений всех перечисленных критериев и интересов всех участников инвестиционного проекта.
\begin{center}
\textbf{\textit{Максимальный денежный отток с учетом дисконтирования}}
\end{center}
\textbf{\textit{Максимальный денежный отток с учетом дисконтирования}} (потребность в финансировании с учётом дисконта, ДПФ) – максимальное значение абсолютной величины отрицательного накопленного дисконтированного сальдо от инвестиционной и операционной деятельности. Величина
ДПФ показывает минимальный дисконтированный объем внешнего (по
отношению к проекту) финансирования проекта, необходимый для обеспечения его финансовой реализуемости.
\begin{center}
	\includegraphics[scale=1]{img2.png} 
\end{center}		
\newpage
 \begin{thebibliography}{9}
 \bibitem{rozenko-2005}
 Михайлова Э. А., Орлова Л. Н. \textit{Экономическая оценка инвестиционных проектов  : учебное пособие}. – Рыбинск: РГАТА, 2008. – 176 с.
 \bibitem{blank}  
    Бланк И.А. \textit{Инвестиционный менеджмент}. - Москва: Эльга, Ника-Центр, 2001. — 448 с.
 \end{thebibliography}
\end{document}